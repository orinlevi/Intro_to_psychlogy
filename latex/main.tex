% !TeX program = xelatex
\documentclass[12pt, a4paper]{book}

\input{preamble}

\begin{document}

% עמוד שער
\begin{titlepage}
\begin{center}
\vspace*{2cm}
{\Huge \textbf{מבוא לפסיכולוגיה}}\\[0.5cm]
{\Large סיכום מקיף לקורס}\\[1cm]
{\large אוניברסיטת תל אביב}\\[0.5cm]
{\large נוצר על ידי Orin Levi}\\[2cm]
{\large \today}
\end{center}
\end{titlepage}

\tableofcontents
\newpage

% ============================================
\section{יחידה 1 - מהי פסיכולוגיה?}
% ============================================

\subsection{הגדרות פסיכולוגיה}

\begin{defbox}
\textbf{הגדרה:} פסיכולוגיה היא המחקר המדעי של התנהגות ותהליכים מנטליים.
\end{defbox}

\begin{itemize}
    \item \textbf{לפי ויקיפדיה}: פסיכולוגיה היא מדע ההתנהגות והתהליכים המנטליים
    \item \textbf{לפי מילון מרים וובסטר}: פסיכולוגיה היא מדע המיינד וההתנהגות
\end{itemize}

\subsection{המדע האמפירי}

\begin{conceptbox}
\textbf{מאפייני השיטה המדעית:}
\begin{itemize}
    \item \textbf{תצפיות ומדידות אובייקטיביות} - הבסיס של המדע האמפירי
    \item \textbf{יכולת ניבוי} - היתרון המשמעותי של המדע האמפירי
    \item \textbf{תיאוריות זמניות} - המדע לא מקבל שום דבר כאמת נצחית
    \item \textbf{שיחזור (רפליקציה)} - חזרה על תוצאות במחקרים שונים
\end{itemize}
\end{conceptbox}

\subsection{דוגמה: טיפול ב-PTSD}

הפרעת PTSD היא הפרעה נפשית שמתפתחת כתוצאה מאירוע טראומטי.

\textbf{סימפטומים עיקריים:}
\begin{itemize}
    \item חוויה מחדש של הטראומה (סיוטים או מחשבות חודרניות)
    \item עלייה כללית בעוררות ובתגובות חרדתיות
    \item ניסיון להימנע מדברים שמזכירים את הטראומה
\end{itemize}

\begin{exbox}
\textbf{השוואת טיפולים:}

\begin{tabular}{|r|r|r|}
\hline
\rowcolor{tableheader} \textcolor{white}{\textbf{טיפול}} & \textcolor{white}{\textbf{תיאור}} & \textcolor{white}{\textbf{יעילות}} \\
\hline
\rowcolor{tablerow1} רלקסציה & תרגילי כיווץ והרפיית שרירים & 40\% \\
\hline
\rowcolor{tablerow2} PE (חשיפה מתמשכת) & חשיפה אינטנסיבית לאירוע & 90\% \\
\hline
\end{tabular}
\end{exbox}

\subsection{מדע בסיסי לעומת מדע יישומי}

\begin{itemize}
    \item \textbf{מדע בסיסי}: מנסה להבין את המנגנון שמסביר תופעה בעולם
    \item \textbf{מדע יישומי}: מפתח שיטות שמשפרות את איכות החיים
\end{itemize}

\begin{notebox}
\textbf{חשוב לזכור:} המדען האמפירי לעולם לא יסתפק בשכל הישר שלו, אלא יתכנן ניסוי אמפירי ואובייקטיבי.
\end{notebox}

\newpage
% ============================================
\section{יחידה 2 - גישות בפסיכולוגיה: הביהביוריזם}
% ============================================

\subsection{מבוא לביהביוריזם}

הזרם המרכזי ששלט בפסיכולוגיה בחלק הראשון של המאה ה-20.

\begin{defbox}
\textbf{הטענה המרכזית:} רק את ההתנהגות אפשר לחקור באופן מדעי, כי רק בהתנהגות אפשר לצפות ולמדוד באופן אובייקטיבי.
\end{defbox}

\begin{warnbox}
\textbf{הקופסה השחורה:} תהליכים מנטליים כמו זיכרון, חשיבה, שפה - אי אפשר לראות ולכן (לטענת הביהביוריסטים) אי אפשר לחקור באופן מדעי.
\end{warnbox}

\subsection{איוון פבלוב - התניה קלאסית}

\begin{exbox}
\textbf{ניסוי הכלבים:}
\begin{enumerate}
    \item צלצול בפעמון לפני מתן אוכל
    \item מדידת כמות הריור בכל פעם
    \item בהדרגה - תגובת הריור לצליל מתחזקת
\end{enumerate}
\end{exbox}

\begin{conceptbox}
\textbf{מושגי יסוד בהתניה קלאסית:}

\begin{tabular}{|r|r|r|}
\hline
\rowcolor{tableheader} \textcolor{white}{\textbf{מונח}} & \textcolor{white}{\textbf{הסבר}} & \textcolor{white}{\textbf{דוגמה}} \\
\hline
\rowcolor{tablerow1} גירוי בלתי מותנה & גירוי שיוצר תגובה אוטומטית & אוכל \\
\hline
\rowcolor{tablerow2} תגובה בלתי מותנית & תגובה אוטומטית לגירוי & ריור \\
\hline
\rowcolor{tablerow1} גירוי מותנה & גירוי ניטרלי שנקשר לגירוי בלתי מותנה & צליל \\
\hline
\rowcolor{tablerow2} תגובה מותנית & תגובה לגירוי המותנה & ריור לצליל \\
\hline
\end{tabular}
\end{conceptbox}

\subsection{תהליכי למידה}

\begin{enumerate}
    \item \textbf{רכישה (Acquisition)} - למידת הקשר בין גירוי מותנה לתגובה מותנית
    \item \textbf{הכחדה (Extinction)} - הפסקת הצימוד וירידה בתגובה
    \item \textbf{החלמה ספונטנית (Spontaneous Recovery)} - חזרת התגובה למחרת
\end{enumerate}

\subsection{ב.פ. סקינר - התניה אופרנטית}

\begin{conceptbox}
\textbf{ארבעה סוגי התניה:}

\begin{tabular}{|r|r|r|}
\hline
\rowcolor{tableheader} \textcolor{white}{\textbf{סוג}} & \textcolor{white}{\textbf{מטרה}} & \textcolor{white}{\textbf{מנגנון}} \\
\hline
\rowcolor{tablerow1} חיזוק חיובי & הגברת התנהגות & מתן דבר נעים \\
\hline
\rowcolor{tablerow2} חיזוק שלילי & הגברת התנהגות & הסרת דבר לא נעים \\
\hline
\rowcolor{tablerow1} עונש חיובי & הפחתת התנהגות & מתן דבר לא נעים \\
\hline
\rowcolor{tablerow2} עונש שלילי & הפחתת התנהגות & מניעת דבר נעים \\
\hline
\end{tabular}
\end{conceptbox}

\begin{notebox}
\textbf{יישום קליני:} טיפול בחשיפה הוא ההפך מהימנעות - מאפשר תהליך הכחדה.
\end{notebox}

\subsection{ג'ון גרסיה - ערעור על הביהביוריזם}

\begin{exbox}
\textbf{Taste Aversion (סלידה מטעם):}

חולדות ששתו מים מתוקים + הקרנה שגרמה לבחילות $\rightarrow$ פיתחו סלידה ממים מתוקים.

\textbf{הממצא המפתיע:}
\begin{itemize}
    \item הסלידה נלמדה גם כשעברו שעות בין המים לבחילות!
    \item הסלידה נוצרה גם אחרי ניסיון אחד בלבד!
\end{itemize}
\end{exbox}

\begin{warnbox}
\textbf{ערעור על הביהביוריזם:}
\begin{itemize}
    \item \textbf{טענה ביהביוריסטית}: כל גירוי יכול להיות מקושר לכל תגובה
    \item \textbf{ממצא גרסיה}: יש \textbf{מוכנות ביולוגית} - חולדות למדו לקשר טעם לבחילות (לא אור או צליל)
\end{itemize}
\end{warnbox}

\subsection{טבלת סיכום - חוקרי הביהביוריזם}

\begin{tabular}{|r|r|r|}
\hline
\rowcolor{tableheader} \textcolor{white}{\textbf{חוקר}} & \textcolor{white}{\textbf{תרומה עיקרית}} & \textcolor{white}{\textbf{ניסוי/מושג מפתח}} \\
\hline
\rowcolor{tablerow1} איוון פבלוב & התניה קלאסית & ניסוי הכלבים - ריור לצליל \\
\hline
\rowcolor{tablerow2} ג'ון ווטסון & התניית פחד & אלברט הקטן \\
\hline
\rowcolor{tablerow1} אדוארד תורנדייק & חוק התוצאה & תיבת החידה \\
\hline
\rowcolor{tablerow2} ב.פ. סקינר & התניה אופרנטית & תיבת סקינר, חיזוקים ועונשים \\
\hline
\rowcolor{tablerow1} ג'ון גרסיה & Taste Aversion & מוכנות ביולוגית \\
\hline
\end{tabular}

\newpage
% ============================================
\section{יחידה 3 - קוגניציה}
% ============================================

\subsection{המהפכה הקוגניטיבית}

\begin{defbox}
\textbf{שתי הטענות המרכזיות:}
\begin{enumerate}
    \item מדע הפסיכולוגיה חייב ויכול לחקור באופן מדעי גם תהליכים מנטליים
    \item אי אפשר להסביר התנהגות רק על סמך למידה מהסביבה - יש מרכיב מולד
\end{enumerate}
\end{defbox}

\textbf{מובילי המהפכה:}
\begin{itemize}
    \item \textbf{נועם חומסקי} - בלשן מ-MIT, טען על קיום מנגנון מולד לרכישת שפה (LAD)
    \item \textbf{ג'ורג' מילר} - חקר את גודל זיכרון העבודה ($7\pm2$ פריטים)
\end{itemize}

\subsection{מנגנון הקשב}

\begin{conceptbox}
\textbf{שתי דרכים להפניית קשב:}
\begin{enumerate}
    \item \textbf{הגירוי עצמו} - גירויים בולטים ועוצמתיים ימשכו קשב
    \item \textbf{רצון/ציפיות} - קשב רצוני (תופעת מסיבת הקוקטייל)
\end{enumerate}
\end{conceptbox}

\subsection{ייצוג (Representation)}

\begin{tabular}{|r|r|r|}
\hline
\rowcolor{tableheader} \textcolor{white}{\textbf{תהליך}} & \textcolor{white}{\textbf{מקור}} & \textcolor{white}{\textbf{דוגמה}} \\
\hline
\rowcolor{tablerow1} Bottom-up & הגירוי עצמו & מה שנכנס דרך החושים \\
\hline
\rowcolor{tablerow2} Top-down & ידע, זיכרון, ציפיות & פרשנות שמשפיעה על הייצוג \\
\hline
\end{tabular}

\subsection{זיכרון מובנה (Constructive Memory)}

\begin{exbox}
\textbf{הניסוי של אליזבת לופטוס (1974):}

נבדקים צפו בסרט תאונה ונשאלו על המהירות.

\textbf{ממצא:} שאלה עם "נמחצה" הובילה להערכת מהירות גבוהה יותר + "זיכרון" של זכוכית שבורה שלא הייתה.

\textbf{השלכות:} לאופן ניסוח השאלה יש השפעה עצומה על דיווח העדים.
\end{exbox}

\subsection{Semantic Priming (הטרמה סמנטית)}

\begin{defbox}
\textbf{הגדרה:} חשיפה למילה אחת משפיעה על עיבוד מילים הקשורות אליה סמנטית.
\end{defbox}

\begin{exbox}
\textbf{Lexical Decision Task:}
\begin{itemize}
    \item המילה "תפוח" מוצגת תחילה (Prime)
    \item המילה "אגס" מזוהה מהר יותר כמילה אמיתית (Target)
    \item הסבר: ייצוג "תפוח" הפעיל מושגים קשורים כמו "פרי", "אגס"
\end{itemize}

\textbf{חשוב:} זוהי מטלה קוגניטיבית, לא ביהביוריסטית!
\end{exbox}

\subsection{טבלת סיכום - חוקרי הקוגניציה}

\begin{tabular}{|r|r|}
\hline
\rowcolor{tableheader} \textcolor{white}{\textbf{חוקר}} & \textcolor{white}{\textbf{תרומה עיקרית}} \\
\hline
\rowcolor{tablerow1} נועם חומסקי & LAD - מנגנון מולד לרכישת שפה \\
\hline
\rowcolor{tablerow2} ג'ורג' מילר & גודל זיכרון העבודה ($7\pm2$) \\
\hline
\rowcolor{tablerow1} דונלד ברודבנט & מודל הסינון בקשב סלקטיבי \\
\hline
\rowcolor{tablerow2} אן טרייסמן & עיבוד מופחת לגירויים לא רלוונטיים \\
\hline
\rowcolor{tablerow1} מייקל פוזנר & קשב רצוני vs אוטומטי \\
\hline
\rowcolor{tablerow2} סטיבן קוסלין & דימויים מנטליים \\
\hline
\rowcolor{tablerow1} אליזבת לופטוס & זיכרון מובנה ועדות שווא \\
\hline
\rowcolor{tablerow2} דניאל כהנמן & הטיות קוגניטיביות \\
\hline
\end{tabular}

\newpage
% ============================================
\section{יחידה 4 - גוף ומוח}
% ============================================

\subsection{הקשר בין גוף ונפש}

\begin{itemize}
    \item \textbf{רדוקציוניזם}: התיאור הביולוגי של תהליכים פסיכולוגיים
    \item \textbf{קשר דו-כיווני}: המוח יוצר מחשבות, והמחשבות משנות את המוח
\end{itemize}

\subsection{הנוירון}

\begin{conceptbox}
\textbf{מבנה הנוירון:}

\begin{tabular}{|r|r|}
\hline
\rowcolor{tableheader} \textcolor{white}{\textbf{חלק}} & \textcolor{white}{\textbf{תפקיד}} \\
\hline
\rowcolor{tablerow1} דנדריטים & קבלת מידע מנוירונים אחרים \\
\hline
\rowcolor{tablerow2} גוף התא & עיבוד המידע \\
\hline
\rowcolor{tablerow1} אקסון & העברת הסיגנל החשמלי \\
\hline
\rowcolor{tablerow2} כפתורים טרמינליים & שחרור נוירוטרנסמיטרים \\
\hline
\end{tabular}
\end{conceptbox}

\subsection{אזורי מוח חשובים}

\begin{itemize}
    \item \textbf{אזור ברוקה} (פול ברוקה, 1861) - פגיעה גורמת לבעיות בהפקת שפה
    \item \textbf{אזור ורניקה} (קרל ורניקה, 1874) - פגיעה גורמת לדיבור שוטף אך חסר משמעות
\end{itemize}

\subsection{המקרה של H.M}

\begin{exbox}
\textbf{רקע:} H.M עבר ניתוח להסרת ההיפוקמפוס משני הצדדים בגלל אפילפסיה.

\textbf{תוצאות:}
\begin{itemize}
    \item \textbf{נשמר:} אינטליגנציה, עיבוד מידע, קשב, זיכרון לטווח קצר
    \item \textbf{נפגע:} יכולת ליצור זיכרונות חדשים (Anterograde Amnesia)
\end{itemize}
\end{exbox}

\begin{conceptbox}
\textbf{סוגי זיכרון ותלות בהיפוקמפוס:}

\begin{tabular}{|r|r|c|}
\hline
\rowcolor{tableheader} \textcolor{white}{\textbf{סוג זיכרון}} & \textcolor{white}{\textbf{תיאור}} & \textcolor{white}{\textbf{תלוי בהיפוקמפוס?}} \\
\hline
\rowcolor{tablerow1} זיכרון לטווח קצר & שניות-דקות, מוגבל & \xmark \\
\hline
\rowcolor{tablerow2} זיכרון דקלרטיבי & אירועים ועובדות & \cmark \\
\hline
\rowcolor{tablerow1} זיכרון פרוצדורלי & מוטורי (רכיבה על אופניים) & \xmark \\
\hline
\rowcolor{tablerow2} התניות & התניית פחד & \xmark{} (אמיגדלה) \\
\hline
\end{tabular}
\end{conceptbox}

\subsection{ניסויי Split-Brain (רוג'ר ספרי)}

\begin{exbox}
\textbf{דוגמה למבחן:}

טל עברה ניתוח Split-Brain. הוצגה לה "מזלג" בשדה ימין ו"כדור" בשדה שמאל.

\begin{itemize}
    \item אם יבקשו ממנה להרים ביד \textbf{ימין} $\rightarrow$ תרים \textbf{מזלג}
    \item אם יבקשו ממנה \textbf{לומר} מה ראתה $\rightarrow$ תגיד \textbf{"מזלג"}
    \item אם יבקשו ממנה להרים ביד \textbf{שמאל} $\rightarrow$ תרים \textbf{כדור}
\end{itemize}

\textbf{הסבר:} שדה ימין $\rightarrow$ המיספרה שמאל (שפה). שדה שמאל $\rightarrow$ המיספרה ימין (ללא שפה, שולטת ביד שמאל).
\end{exbox}

\subsection{שיטות לחקר המוח}

\begin{tabular}{|r|r|c|}
\hline
\rowcolor{tableheader} \textcolor{white}{\textbf{שיטה}} & \textcolor{white}{\textbf{מה עושה}} & \textcolor{white}{\textbf{מודדת או משפיעה?}} \\
\hline
\rowcolor{tablerow1} fMRI & מודדת שינויים בצריכת חמצן & מודדת \\
\hline
\rowcolor{tablerow2} EEG & מודדת פעילות חשמלית & מודדת \\
\hline
\rowcolor{tablerow1} TMS & משפיעה על פעילות באמצעות שדה מגנטי & \textbf{משפיעה!} \\
\hline
\end{tabular}

\begin{warnbox}
\textbf{TMS - גרייה מגנטית:} TMS (Transcranial Magnetic Stimulation) משתמשת בשדה מגנטי כדי \textbf{להשפיע} על פעילות נוירונים במוח - לא רק למדוד!
\end{warnbox}

\subsection{הייצוג הסומטוסנסורי}

\begin{notebox}
\textbf{חשוב למבחן:} הייצוג במוח נקבע לפי \textbf{רגישות}, לא לפי גודל פיזי!

לשפתיים יש שטח ייצוג \textbf{גדול יותר} מהגב - כי השפתיים \textbf{רגישות יותר} למגע.
\end{notebox}

\subsection{טבלת סיכום - חוקרי המוח}

\begin{tabular}{|r|r|r|}
\hline
\rowcolor{tableheader} \textcolor{white}{\textbf{חוקר}} & \textcolor{white}{\textbf{תרומה עיקרית}} & \textcolor{white}{\textbf{שנה}} \\
\hline
\rowcolor{tablerow1} רנה דקארט & דואליזם, בלוטת האצטרובל & המאה ה-17 \\
\hline
\rowcolor{tablerow2} פרנץ גאל & פרנולוגיה (נדחתה) & המאה ה-19 \\
\hline
\rowcolor{tablerow1} פול ברוקה & אזור ברוקה - הפקת שפה & 1861 \\
\hline
\rowcolor{tablerow2} קרל ורניקה & אזור ורניקה - הבנת שפה & 1874 \\
\hline
\rowcolor{tablerow1} רוג'ר ספרי & ניסויי Split-Brain & נובל 1981 \\
\hline
\rowcolor{tablerow2} ברנדה מילנר & מקרה H.M, זיכרון פרוצדורלי & שנות ה-50 \\
\hline
\end{tabular}

\newpage
% ============================================
\section{יחידה 5 - פסיכולוגיה התפתחותית}
% ============================================

\subsection{התקופה הקריטית}

\begin{defbox}
\textbf{הגדרה:} התקופה שבה החשיפה לסביבה אמורה להתרחש להתפתחות מלאה.
\end{defbox}

\begin{exbox}
\textbf{דוגמה - ג'יני:} ילדה שלא נחשפה לשפה עד גיל 13 - לא הצליחה ללמוד מעבר לרמה בסיסית.
\end{exbox}

\subsection{איך חוקרים מיינד של תינוקות?}

\begin{itemize}
    \item \textbf{קצב דופק} - שינוי בדופק לגירוי
    \item \textbf{תדירות מציצה} - שינוי בקצב מציצת מוצץ
    \item \textbf{זמן הסתכלות} - כמה זמן מסתכלים על גירוי (הנפוץ ביותר)
\end{itemize}

\subsection{Perceptual Narrowing}

\begin{conceptbox}
\begin{itemize}
    \item \textbf{עד 6 חודשים}: מבחינים בכל הפונמות בעולם ופנים מכל הגזעים
    \item \textbf{מגיל 9 חודשים}: מבחינים רק בפונמות של שפת האם ובפנים מוכרים
\end{itemize}
\end{conceptbox}

\subsection{תיאוריית ההתפתחות של פיאז'ה}

\begin{tabular}{|r|c|r|}
\hline
\rowcolor{tableheader} \textcolor{white}{\textbf{שלב}} & \textcolor{white}{\textbf{גיל}} & \textcolor{white}{\textbf{מאפיינים}} \\
\hline
\rowcolor{tablerow1} סנסורי-מוטורי & 0-2 & קביעות אובייקט \\
\hline
\rowcolor{tablerow2} פרה-אופרציונלי & 2-7 & אגוצנטריות, שפה סימבולית \\
\hline
\rowcolor{tablerow1} אופרציונלי קונקרטי & 7-10 & הבנת שימור \\
\hline
\rowcolor{tablerow2} אופרציונלי פורמלי & 10+ & חשיבה אבסטרקטית \\
\hline
\end{tabular}

\subsection{Theory of Mind}

\begin{defbox}
\textbf{הגדרה:} היכולת להבין שלאנשים אחרים יש כוונות, אמונות ורצונות שונים משלי.

מתפתחת סביב גיל 4-5 (נבדקת במשימת False Belief).
\end{defbox}

\newpage
% ============================================
\section{יחידה 6 - פסיכולוגיה חברתית}
% ============================================

\subsection{השפעה על ביצועים}

\begin{tabular}{|r|r|}
\hline
\rowcolor{tableheader} \textcolor{white}{\textbf{תופעה}} & \textcolor{white}{\textbf{מתי}} \\
\hline
\rowcolor{tablerow1} הקלה חברתית & משימה קלה - ביצוע משתפר \\
\hline
\rowcolor{tablerow2} עכבה חברתית & משימה קשה - ביצוע נפגע \\
\hline
\end{tabular}

\subsubsection{הניסוי של רוברט זאיונק (על ג'וקים!)}

\begin{exbox}
\begin{itemize}
    \item \textbf{משימה קלה} (מסלול ישר): ביצוע טוב יותר בנוכחות אחרים
    \item \textbf{משימה קשה} (מסלול צלב): ביצוע טוב יותר לבד
\end{itemize}

\textbf{ההסבר:} עוררות ומוטיבציה עוזרות במשימות קלות אך מפריעות במשימות קשות.

\textbf{המשמעות:} התופעה קיימת גם בג'וקים ובחולדות - זו תופעה בסיסית ואוטומטית שלא דורשת קוגניציה גבוהה!
\end{exbox}

\subsubsection{האם הנוכחות או המבט?}

\begin{tabular}{|r|r|}
\hline
\rowcolor{tableheader} \textcolor{white}{\textbf{תנאי}} & \textcolor{white}{\textbf{ביצוע}} \\
\hline
\rowcolor{tablerow1} לבד & בסיסי \\
\hline
\rowcolor{tablerow2} נוכחות אחרים עם עיניים \textbf{מכוסות} & כמו לבד! \\
\hline
\rowcolor{tablerow1} נוכחות אחרים עם עיניים \textbf{פתוחות} & השפעה על הביצוע \\
\hline
\end{tabular}

\begin{notebox}
\textbf{מסקנה:} לא הנוכחות עצמה משפיעה, אלא \textbf{המבט} - העובדה שאנשים מתבוננים ויכולים להעריך!
\end{notebox}

\subsection{קונפורמיזם - הניסוי של סלומון אש}

\begin{exbox}
\textbf{הניסוי:} השוואת אורך קווים (תשובה אובייקטיבית!) כאשר משתפי פעולה עונים תשובות שגויות.

\textbf{תוצאות:}
\begin{itemize}
    \item 76\% ענו לפחות תשובה אחת שגויה
    \item 50\% ענו שגוי בלפחות מחצית מהסבבים
\end{itemize}
\end{exbox}

\begin{notebox}
\textbf{חשוב:} מספיק שאדם אחד בקבוצה יאמר תשובה שונה - והנבדק יאמר את התשובה הנכונה.
\end{notebox}

\subsection{אפקט הצופה מהצד}

ככל שיש יותר אנשים - פחות סיכוי לעזור.

\textbf{הסברים:}
\begin{enumerate}
    \item \textbf{Pluralistic Ignorance} - "אם אחרים לא עוזרים, המצב לא מסוכן"
    \item \textbf{פיזור אחריות} - "מישהו אחר יטפל"
    \item \textbf{Audience Inhibition} - חשש מלהביך את עצמי
\end{enumerate}

\subsection{ניסוי מילגרם (1963)}

\begin{warnbox}
\textbf{תוצאות מזעזעות:} 60\%+ הסכימו לתת שוקים מסוכנים לאדם אחר כשנאמר להם לעשות זאת.
\end{warnbox}

\begin{defbox}
\textbf{טעות הייחוס בסיסית:} נטייה לייחס התנהגות של אחרים לאישיותם, ולא למצב שבו הם נמצאים.
\end{defbox}

\subsection{Stereotype Threat}

\begin{defbox}
\textbf{הגדרה:} חשש שביצועים יאוששו סטריאוטיפ שלילי.

\textbf{דוגמה:} נשים במתמטיקה - כשמזכירים הבדלים מגדריים, הביצוע יורד.
\end{defbox}

\subsection{טבלת סיכום - ניסויי פסיכולוגיה חברתית}

\begin{tabular}{|r|r|r|}
\hline
\rowcolor{tableheader} \textcolor{white}{\textbf{חוקר}} & \textcolor{white}{\textbf{ניסוי/תופעה}} & \textcolor{white}{\textbf{ממצא מרכזי}} \\
\hline
\rowcolor{tablerow1} נורמן טריפלט (1898) & גלגול חכה & ביצוע טוב יותר בנוכחות אחרים \\
\hline
\rowcolor{tablerow2} רוברט זאיונק & ג'וקים במסלול & קושי המשימה קובע אם נוכחות עוזרת \\
\hline
\rowcolor{tablerow1} מוזאפר שריף & אפקט אוטוקינטי & השפעה אינפורמטיבית \\
\hline
\rowcolor{tablerow2} סלומון אש & אורך קווים & קונפורמיזם גם בשיפוט אובייקטיבי \\
\hline
\rowcolor{tablerow1} סטנלי מילגרם & שוקים חשמליים & 60\%+ צייתו לסמכות \\
\hline
\rowcolor{tablerow2} לטנה ודארלי & אפקט הצופה & יותר אנשים = פחות עוזרים \\
\hline
\end{tabular}

\begin{warnbox}
\textbf{מה משותף למילגרם ואש?}

בשני הניסויים: \textbf{נוכחות של אדם שמורד/מציג דעה שונה - צמצמה את ההשפעה החברתית!}
\end{warnbox}

\newpage
% ============================================
\section{יחידה 7 - הבדלים בינאישיים}
% ============================================

\subsection{מבחני אינטליגנציה}

\begin{tabular}{|r|r|}
\hline
\rowcolor{tableheader} \textcolor{white}{\textbf{מאפיין}} & \textcolor{white}{\textbf{הסבר}} \\
\hline
\rowcolor{tablerow1} מהימנות & תוצאה דומה במדידות חוזרות \\
\hline
\rowcolor{tablerow2} תוקף & מודד את מה שאמור למדוד \\
\hline
\end{tabular}

\subsection{תיאוריית G (ספירמן)}

\begin{itemize}
    \item \textbf{G} = יכולת כללית משותפת לכל היכולות
    \item \textbf{S} = יכולות ספציפיות שאינן קשורות זו לזו
\end{itemize}

\subsection{חמשת הגדולים (Big 5)}

\begin{conceptbox}
\begin{tabular}{|r|r|}
\hline
\rowcolor{tableheader} \textcolor{white}{\textbf{תכונה}} & \textcolor{white}{\textbf{מאפיינים}} \\
\hline
\rowcolor{tablerow1} מוחצנות (Extroversion) & חברותיות, דברנות, פעלתנות \\
\hline
\rowcolor{tablerow2} יציבות רגשית (Neuroticism) & חרדה, דיכאון, עצבנות \\
\hline
\rowcolor{tablerow1} נועם הליכות (Agreeableness) & אדיבות, אמון, מזג נוח \\
\hline
\rowcolor{tablerow2} מצפוניות (Conscientiousness) & אחריות, יסודיות, משמעת עצמית \\
\hline
\rowcolor{tablerow1} פתיחות (Openness) & דמיון, סקרנות, אופקים רחבים \\
\hline
\end{tabular}
\end{conceptbox}

\subsection{מבחן המרשמלו (וולטר מישל)}

\begin{exbox}
\textbf{הניסוי:} ילדים יכולים לאכול מרשמלו עכשיו או לחכות ולקבל עוד אחד.

\textbf{מעקב לאורך שנים:} ילדים שהצליחו לחכות הפכו לבני נוער עם יכולת ריכוז גבוהה יותר, שליטה עצמית, וציונים גבוהים יותר.
\end{exbox}

\subsection{פרשנות מחקרי תאומים}

\begin{tabular}{|r|r|}
\hline
\rowcolor{tableheader} \textcolor{white}{\textbf{ממצא}} & \textcolor{white}{\textbf{פרשנות}} \\
\hline
\rowcolor{tablerow1} MZ $>$ DZ & תורשה משפיעה על התכונה \\
\hline
\rowcolor{tablerow2} MZ יחד $>$ MZ נפרד & סביבה משותפת משפיעה \\
\hline
\rowcolor{tablerow1} MZ $<$ 100\% & סביבה משפיעה (לא רק גנים) \\
\hline
\rowcolor{tablerow2} MZ נפרד דומים מאוד & תורשה חזקה מאוד \\
\hline
\end{tabular}

\begin{notebox}
\textbf{הבדל בין אינטליגנציה לאישיות:}

\textbf{אינטליגנציה:} השפעת \textbf{תורשה} חזקה יותר מסביבה.

\textbf{אישיות:} השפעת \textbf{סביבה לא-משותפת} (חברים, חוויות אישיות) חזקה יותר מסביבה משותפת.
\end{notebox}

\newpage
% ============================================
\section{יחידה 8 - פסיכופתולוגיה}
% ============================================

\subsection{ה-DSM}

\begin{defbox}
\textbf{DSM} = Diagnostic and Statistical Manual of Mental Disorders

ספר שמאגד את כל ההפרעות המנטליות והסימפטומים שמאפיינים אותן.
\end{defbox}

\textbf{קטגוריות הפרעות:}
\begin{itemize}
    \item הפרעות נוירו-התפתחותיות (אוטיזם, פיגור)
    \item סכיזופרניה
    \item הפרעות דיכאון
    \item הפרעות חרדה
    \item OCD
    \item הפרעות טראומה
    \item התמכרויות
    \item הפרעות אישיות
\end{itemize}

\subsection{סוגי טיפול}

\subsubsection{הטיפול הביולוגי}

\begin{tabular}{|r|r|}
\hline
\rowcolor{tableheader} \textcolor{white}{\textbf{סוג}} & \textcolor{white}{\textbf{שימוש}} \\
\hline
\rowcolor{tablerow1} ליתיום & הפרעה דו-קוטבית \\
\hline
\rowcolor{tablerow2} אנטי-פסיכוטיות & סכיזופרניה \\
\hline
\rowcolor{tablerow1} מרגיעות & חרדה \\
\hline
\rowcolor{tablerow2} נוגדות דיכאון & דיכאון (פרוזק) \\
\hline
\end{tabular}

\subsubsection{הטיפול הפסיכודינמי}

המקור לפתולוגיה: קונפליקט בין מודע ללא מודע.

\subsubsection{הטיפול הקוגניטיבי-התנהגותי (CBT)}

\begin{conceptbox}
\begin{itemize}
    \item \textbf{החלק ההתנהגותי (וולפה)}: הקהיה שיטתית - חשיפה הדרגתית לגירויים מפחידים
    \item \textbf{החלק הקוגניטיבי (אהרון בק)}: זיהוי הטיות קוגניטיביות ובחינת תוקפן
\end{itemize}
\end{conceptbox}

\begin{summarybox}
\textbf{מאפייני CBT:}
\begin{itemize}
    \item מטרות מוגדרות
    \item טיפול קצר וממוקד
    \item מטפל אקטיבי ומכוון
    \item שיעורי בית
    \item Psycho-education
\end{itemize}
\end{summarybox}

\subsubsection{השוואה: CBT לעומת ACT}

\begin{tabular}{|r|r|r|}
\hline
\rowcolor{tableheader} \textcolor{white}{\textbf{היבט}} & \textcolor{white}{\textbf{CBT}} & \textcolor{white}{\textbf{ACT}} \\
\hline
\rowcolor{tablerow1} מטרה & שינוי מחשבות שליליות & קבלת מחשבות ללא שיפוט \\
\hline
\rowcolor{tablerow2} יחס למחשבות & מחשבות יכולות להיות שגויות & מחשבות הן רק מחשבות \\
\hline
\rowcolor{tablerow1} טכניקה מרכזית & אתגור קוגניטיבי & Defusion (ניתוק) \\
\hline
\rowcolor{tablerow2} גישה לערכים & פחות מרכזי & מרכזי - פעולה לפי ערכים \\
\hline
\end{tabular}

% ============================================
\vfill
\begin{center}
\rule{0.5\textwidth}{0.5pt}\\[0.3cm]
\textit{סיכום זה נוצר באמצעות Claude AI}\\
\url{https://github.com/orinlevi/Intro_to_psychlogy}
\end{center}

\end{document}
