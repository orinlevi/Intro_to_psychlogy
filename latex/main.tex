% !TeX program = xelatex
\documentclass[12pt, a4paper]{book}

\input{preamble}

\begin{document}

% עמוד שער
\begin{titlepage}
\begin{center}
\vspace*{2cm}
{\Huge \textbf{מבוא לפסיכולוגיה}}\\[0.5cm]
{\Large סיכום מקיף לקורס}\\[1cm]
{\large אוניברסיטת תל אביב}\\[0.5cm]
{\large נוצר על ידי Orin Levi}\\[2cm]
{\large \today}
\end{center}
\end{titlepage}

\tableofcontents
\newpage

% ============================================
\section{יחידה 1 - מהי פסיכולוגיה?}
% ============================================

\subsection{הגדרות פסיכולוגיה}

\begin{defbox}
\textbf{הגדרה:} פסיכולוגיה היא המחקר המדעי של התנהגות ותהליכים מנטליים.
\end{defbox}

\begin{itemize}
    \item \textbf{לפי ויקיפדיה}: פסיכולוגיה היא מדע ההתנהגות והתהליכים המנטליים
    \item \textbf{לפי מילון מרים וובסטר}: פסיכולוגיה היא מדע המיינד וההתנהגות
\end{itemize}

\subsection{המדע האמפירי}

\begin{conceptbox}
\textbf{מאפייני השיטה המדעית:}
\begin{itemize}
    \item \textbf{תצפיות ומדידות אובייקטיביות} - הבסיס של המדע האמפירי
    \item \textbf{יכולת ניבוי} - היתרון המשמעותי של המדע האמפירי
    \item \textbf{תיאוריות זמניות} - המדע לא מקבל שום דבר כאמת נצחית
    \item \textbf{שיחזור (רפליקציה)} - חזרה על תוצאות במחקרים שונים
\end{itemize}
\end{conceptbox}

\subsection{דוגמה: טיפול ב-PTSD}

הפרעת PTSD היא הפרעה נפשית שמתפתחת כתוצאה מאירוע טראומטי.

\textbf{סימפטומים עיקריים:}
\begin{itemize}
    \item חוויה מחדש של הטראומה (סיוטים או מחשבות חודרניות)
    \item עלייה כללית בעוררות ובתגובות חרדתיות
    \item ניסיון להימנע מדברים שמזכירים את הטראומה
\end{itemize}

\begin{exbox}
\textbf{השוואת טיפולים:}

\begin{tabular}{|r|r|r|}
\hline
\rowcolor{tableheader} \textcolor{white}{\textbf{טיפול}} & \textcolor{white}{\textbf{תיאור}} & \textcolor{white}{\textbf{יעילות}} \\
\hline
\rowcolor{tablerow1} רלקסציה & תרגילי כיווץ והרפיית שרירים & 40\% \\
\hline
\rowcolor{tablerow2} PE (חשיפה מתמשכת) & חשיפה אינטנסיבית לאירוע & 90\% \\
\hline
\end{tabular}
\end{exbox}

\subsection{מדע בסיסי לעומת מדע יישומי}

\begin{itemize}
    \item \textbf{מדע בסיסי}: מנסה להבין את המנגנון שמסביר תופעה בעולם
    \item \textbf{מדע יישומי}: מפתח שיטות שמשפרות את איכות החיים
\end{itemize}

\begin{notebox}
\textbf{חשוב לזכור:} המדען האמפירי לעולם לא יסתפק בשכל הישר שלו, אלא יתכנן ניסוי אמפירי ואובייקטיבי.
\end{notebox}

\newpage
% ============================================
\section{יחידה 2 - גישות בפסיכולוגיה: הביהביוריזם}
% ============================================

\subsection{מבוא לביהביוריזם}

הזרם המרכזי ששלט בפסיכולוגיה בחלק הראשון של המאה ה-20.

\begin{defbox}
\textbf{הטענה המרכזית:} רק את ההתנהגות אפשר לחקור באופן מדעי, כי רק בהתנהגות אפשר לצפות ולמדוד באופן אובייקטיבי.
\end{defbox}

\begin{warnbox}
\textbf{הקופסה השחורה:} תהליכים מנטליים כמו זיכרון, חשיבה, שפה - אי אפשר לראות ולכן (לטענת הביהביוריסטים) אי אפשר לחקור באופן מדעי.
\end{warnbox}

\subsection{איוון פבלוב - התניה קלאסית}

\begin{exbox}
\textbf{ניסוי הכלבים:}
\begin{enumerate}
    \item צלצול בפעמון לפני מתן אוכל
    \item מדידת כמות הריור בכל פעם
    \item בהדרגה - תגובת הריור לצליל מתחזקת
\end{enumerate}
\end{exbox}

\begin{conceptbox}
\textbf{מושגי יסוד בהתניה קלאסית:}

\begin{tabular}{|r|r|r|}
\hline
\rowcolor{tableheader} \textcolor{white}{\textbf{מונח}} & \textcolor{white}{\textbf{הסבר}} & \textcolor{white}{\textbf{דוגמה}} \\
\hline
\rowcolor{tablerow1} גירוי בלתי מותנה & גירוי שיוצר תגובה אוטומטית & אוכל \\
\hline
\rowcolor{tablerow2} תגובה בלתי מותנית & תגובה אוטומטית לגירוי & ריור \\
\hline
\rowcolor{tablerow1} גירוי מותנה & גירוי ניטרלי שנקשר לגירוי בלתי מותנה & צליל \\
\hline
\rowcolor{tablerow2} תגובה מותנית & תגובה לגירוי המותנה & ריור לצליל \\
\hline
\end{tabular}
\end{conceptbox}

\subsection{תהליכי למידה}

\begin{enumerate}
    \item \textbf{רכישה (Acquisition)} - למידת הקשר בין גירוי מותנה לתגובה מותנית
    \item \textbf{הכחדה (Extinction)} - הפסקת הצימוד וירידה בתגובה
    \item \textbf{החלמה ספונטנית (Spontaneous Recovery)} - חזרת התגובה למחרת
\end{enumerate}

\subsection{ב.פ. סקינר - התניה אופרנטית}

\subsubsection{תיבת סקינר (Skinner Box)}

\begin{exbox}
סקינר פיתח את \textbf{תיבת סקינר} - כלוב מיוחד שבו חקר את חוקי ההתניה האופרנטית:
\begin{itemize}
    \item השתמש באוכל ושתייה כפרס
    \item שוק חשמלי (לא מזיק) כעונש
    \item יונים וחולדות לחצו על מנוף ולמדו את הקשר בין פעולה לתוצאה
\end{itemize}
\end{exbox}

\subsubsection{עיצוב (Shaping)}

\begin{defbox}
\textbf{עיצוב (Shaping)} = חיזוק התנהגויות שמתקרבות להתנהגות הרצויה, בהדרגה.

\textbf{איך זה עובד?}
\begin{enumerate}
    \item בהתחלה - מחזקים כל התנהגות שמתקרבת למטרה
    \item בהדרגה - מחמירים את הקריטריון
    \item בסוף - מחזקים רק את ההתנהגות המדויקת הרצויה
\end{enumerate}
\end{defbox}

\begin{exbox}
\textbf{יונים משחקות פינג-פונג!}

סקינר הצליח ללמד \textbf{יונים לשחק פינג-פונג} באמצעות שיטת העיצוב! כל פעולה קטנה שהתקרבה להתנהגות הרצויה קיבלה חיזוק, עד שהיונים למדו לשחק משחק שלם.
\end{exbox}

\begin{conceptbox}
\textbf{ארבעה סוגי התניה:}

\begin{tabular}{|r|r|r|}
\hline
\rowcolor{tableheader} \textcolor{white}{\textbf{סוג}} & \textcolor{white}{\textbf{מטרה}} & \textcolor{white}{\textbf{מנגנון}} \\
\hline
\rowcolor{tablerow1} חיזוק חיובי & הגברת התנהגות & מתן דבר נעים \\
\hline
\rowcolor{tablerow2} חיזוק שלילי & הגברת התנהגות & הסרת דבר לא נעים \\
\hline
\rowcolor{tablerow1} עונש חיובי & הפחתת התנהגות & מתן דבר לא נעים \\
\hline
\rowcolor{tablerow2} עונש שלילי & הפחתת התנהגות & מניעת דבר נעים \\
\hline
\end{tabular}
\end{conceptbox}

\begin{notebox}
\textbf{יישום קליני:} טיפול בחשיפה הוא ההפך מהימנעות - מאפשר תהליך הכחדה.
\end{notebox}

\subsection{ג'ון גרסיה - ערעור על הביהביוריזם}

\begin{exbox}
\textbf{Taste Aversion (סלידה מטעם):}

חולדות ששתו מים מתוקים + הקרנה שגרמה לבחילות $\rightarrow$ פיתחו סלידה ממים מתוקים.

\textbf{הממצא המפתיע:}
\begin{itemize}
    \item הסלידה נלמדה גם כשעברו שעות בין המים לבחילות!
    \item הסלידה נוצרה גם אחרי ניסיון אחד בלבד!
\end{itemize}
\end{exbox}

\begin{warnbox}
\textbf{ערעור על הביהביוריזם:}
\begin{itemize}
    \item \textbf{טענה ביהביוריסטית}: כל גירוי יכול להיות מקושר לכל תגובה
    \item \textbf{ממצא גרסיה}: יש \textbf{מוכנות ביולוגית} - חולדות למדו לקשר טעם לבחילות (לא אור או צליל)
\end{itemize}
\end{warnbox}

\subsection{טבלת סיכום - חוקרי הביהביוריזם}

\begin{tabular}{|r|r|r|}
\hline
\rowcolor{tableheader} \textcolor{white}{\textbf{חוקר}} & \textcolor{white}{\textbf{תרומה עיקרית}} & \textcolor{white}{\textbf{ניסוי/מושג מפתח}} \\
\hline
\rowcolor{tablerow1} איוון פבלוב & התניה קלאסית & ניסוי הכלבים - ריור לצליל \\
\hline
\rowcolor{tablerow2} ג'ון ווטסון & התניית פחד & אלברט הקטן \\
\hline
\rowcolor{tablerow1} אדוארד תורנדייק & חוק התוצאה & תיבת החידה \\
\hline
\rowcolor{tablerow2} ב.פ. סקינר & התניה אופרנטית & תיבת סקינר, חיזוקים ועונשים \\
\hline
\rowcolor{tablerow1} ג'ון גרסיה & Taste Aversion & מוכנות ביולוגית \\
\hline
\end{tabular}

\newpage
% ============================================
\section{יחידה 3 - קוגניציה}
% ============================================

\subsection{המהפכה הקוגניטיבית}

\begin{defbox}
\textbf{שתי הטענות המרכזיות:}
\begin{enumerate}
    \item מדע הפסיכולוגיה חייב ויכול לחקור באופן מדעי גם תהליכים מנטליים
    \item אי אפשר להסביר התנהגות רק על סמך למידה מהסביבה - יש מרכיב מולד
\end{enumerate}
\end{defbox}

\textbf{מובילי המהפכה:}
\begin{itemize}
    \item \textbf{נועם חומסקי} - בלשן מ-MIT, טען על קיום מנגנון מולד לרכישת שפה (LAD)
    \item \textbf{ג'ורג' מילר} - חקר את גודל זיכרון העבודה ($7\pm2$ פריטים)
\end{itemize}

\subsection{מנגנון הקשב}

\begin{conceptbox}
\textbf{שתי דרכים להפניית קשב:}
\begin{enumerate}
    \item \textbf{הגירוי עצמו} - גירויים בולטים ועוצמתיים ימשכו קשב
    \item \textbf{רצון/ציפיות} - קשב רצוני (תופעת מסיבת הקוקטייל)
\end{enumerate}
\end{conceptbox}

\subsubsection{תופעת מסיבת הקוקטייל (Cocktail Party Effect)}

\begin{defbox}
\textbf{תופעת מסיבת הקוקטייל:} היכולת לשים לב לשיחה אחת בסביבה רועשת, למרות רעשי רקע רבים.

\textbf{מה מפר את הסינון?}
\begin{itemize}
    \item שמיעת השם שלך (קשב אוטומטי למידע רלוונטי אישית)
    \item זיהוי מילים רגשיות או חשובות
\end{itemize}
\end{defbox}

\subsubsection{מטלת Dichotic Listening}

\begin{exbox}
\textbf{מטלת הקשבה דיכוטית:}
\begin{itemize}
    \item באוזן אחת - מסר אחד
    \item באוזן השנייה - מסר אחר
    \item הנבדק מתבקש להקשיב רק לאוזן אחת ולדווח עליה
\end{itemize}

\textbf{מה הנבדקים מזהים מהאוזן ה"מוזנחת"?}
\begin{itemize}
    \item מזהים: גבר/אישה, שפה/צליל
    \item \textbf{לא מזהים}: תוכן סמנטי, שפה, מילים ספציפיות
\end{itemize}
\end{exbox}

\subsubsection{הויכוח: ברודבנט vs טריזמן}

\begin{tabular}{|r|r|r|}
\hline
\rowcolor{tableheader} \textcolor{white}{\textbf{תיאוריה}} & \textcolor{white}{\textbf{חוקר}} & \textcolor{white}{\textbf{טענה}} \\
\hline
\rowcolor{tablerow1} Filter Theory & דונלד ברודבנט & מסננים לחלוטין לפי מאפיינים פיזיים \\
\hline
\rowcolor{tablerow2} Attenuation Theory & אן טריזמן & מנמיכים עוצמה אבל עדיין מעבדים \\
\hline
\end{tabular}

\begin{notebox}
\textbf{מי צודק?} טריזמן נחשבת יותר צודקת - כי אנשים מזהים את השם שלהם גם באוזן ה"מוזנחת".
\end{notebox}

\subsubsection{הניסוי של איגלי, דרייבר וראפאל (Egly, Driver \& Rafal)}

\begin{exbox}
\textbf{שאלת המחקר:} האם קשב מופנה למיקום במרחב או לאובייקט?

\textbf{הממצא המפתיע:}
\begin{itemize}
    \item זמן תגובה מהיר יותר לגירוי \textbf{באותו מלבן} (גם אם המרחק שווה!)
    \item קשב מופנה \textbf{לאובייקט}, לא רק למיקום
\end{itemize}
\end{exbox}

\subsection{ייצוג (Representation)}

\begin{tabular}{|r|r|r|}
\hline
\rowcolor{tableheader} \textcolor{white}{\textbf{תהליך}} & \textcolor{white}{\textbf{מקור}} & \textcolor{white}{\textbf{דוגמה}} \\
\hline
\rowcolor{tablerow1} Bottom-up & הגירוי עצמו & מה שנכנס דרך החושים \\
\hline
\rowcolor{tablerow2} Top-down & ידע, זיכרון, ציפיות & פרשנות שמשפיעה על הייצוג \\
\hline
\end{tabular}

\subsection{זיכרון מובנה (Constructive Memory)}

\begin{exbox}
\textbf{הניסוי של אליזבת לופטוס (1974):}

נבדקים צפו בסרט תאונה ונשאלו על המהירות.

\textbf{ממצא:} שאלה עם "נמחצה" הובילה להערכת מהירות גבוהה יותר + "זיכרון" של זכוכית שבורה שלא הייתה.

\textbf{השלכות:} לאופן ניסוח השאלה יש השפעה עצומה על דיווח העדים.
\end{exbox}

\subsection{Semantic Priming (הטרמה סמנטית)}

\begin{defbox}
\textbf{הגדרה:} חשיפה למילה אחת משפיעה על עיבוד מילים הקשורות אליה סמנטית.
\end{defbox}

\begin{exbox}
\textbf{Lexical Decision Task:}
\begin{itemize}
    \item המילה "תפוח" מוצגת תחילה (Prime)
    \item המילה "אגס" מזוהה מהר יותר כמילה אמיתית (Target)
    \item הסבר: ייצוג "תפוח" הפעיל מושגים קשורים כמו "פרי", "אגס"
\end{itemize}

\textbf{חשוב:} זוהי מטלה קוגניטיבית, לא ביהביוריסטית!
\end{exbox}

\subsection{טבלת סיכום - חוקרי הקוגניציה}

\begin{tabular}{|r|r|}
\hline
\rowcolor{tableheader} \textcolor{white}{\textbf{חוקר}} & \textcolor{white}{\textbf{תרומה עיקרית}} \\
\hline
\rowcolor{tablerow1} נועם חומסקי & LAD - מנגנון מולד לרכישת שפה \\
\hline
\rowcolor{tablerow2} ג'ורג' מילר & גודל זיכרון העבודה ($7\pm2$) \\
\hline
\rowcolor{tablerow1} דונלד ברודבנט & מודל הסינון בקשב סלקטיבי \\
\hline
\rowcolor{tablerow2} אן טרייסמן & עיבוד מופחת לגירויים לא רלוונטיים \\
\hline
\rowcolor{tablerow1} מייקל פוזנר & קשב רצוני vs אוטומטי \\
\hline
\rowcolor{tablerow2} סטיבן קוסלין & דימויים מנטליים \\
\hline
\rowcolor{tablerow1} אליזבת לופטוס & זיכרון מובנה ועדות שווא \\
\hline
\rowcolor{tablerow2} דניאל כהנמן & הטיות קוגניטיביות \\
\hline
\end{tabular}

\newpage
% ============================================
\section{יחידה 4 - גוף ומוח}
% ============================================

\subsection{הקשר בין גוף ונפש}

\begin{itemize}
    \item \textbf{רדוקציוניזם}: התיאור הביולוגי של תהליכים פסיכולוגיים
    \item \textbf{קשר דו-כיווני}: המוח יוצר מחשבות, והמחשבות משנות את המוח
\end{itemize}

\subsection{הנוירון}

\begin{conceptbox}
\textbf{מבנה הנוירון:}

\begin{tabular}{|r|r|}
\hline
\rowcolor{tableheader} \textcolor{white}{\textbf{חלק}} & \textcolor{white}{\textbf{תפקיד}} \\
\hline
\rowcolor{tablerow1} דנדריטים & קבלת מידע מנוירונים אחרים \\
\hline
\rowcolor{tablerow2} גוף התא & עיבוד המידע \\
\hline
\rowcolor{tablerow1} אקסון & העברת הסיגנל החשמלי \\
\hline
\rowcolor{tablerow2} כפתורים טרמינליים & שחרור נוירוטרנסמיטרים \\
\hline
\end{tabular}
\end{conceptbox}

\subsection{אזורי מוח חשובים}

\begin{itemize}
    \item \textbf{אזור ברוקה} (פול ברוקה, 1861) - פגיעה גורמת לבעיות בהפקת שפה
    \item \textbf{אזור ורניקה} (קרל ורניקה, 1874) - פגיעה גורמת לדיבור שוטף אך חסר משמעות
\end{itemize}

\subsection{מחקר נהגי מוניות בלונדון (2000)}

\begin{exbox}
\textbf{מחקר מפורסם מאוניברסיטת לונדון:}

בדקו את המוח של נהגי מוניות לונדוניים שעברו "The Knowledge" - מבחן קשה שדורש לזכור את כל רחובות לונדון ומסלולים (לפני עידן ה-GPS!).

\textbf{הממצא:} ההיפוקמפוס של נהגי המוניות היה \textbf{גדול יותר} מאשר אצל אנשים רגילים!

\textbf{יותר מזה:} ככל ש\textbf{יותר שנים} נהגו במונית $\rightarrow$ ההיפוקמפוס \textbf{גדול יותר}!

\textbf{מסקנה:} המוח משתנה בעקבות למידה - \textbf{נוירופלסטיות}!
\end{exbox}

\subsection{המקרה של H.M}

\begin{exbox}
\textbf{רקע:} H.M עבר ניתוח להסרת ההיפוקמפוס משני הצדדים בגלל אפילפסיה.

\textbf{תוצאות:}
\begin{itemize}
    \item \textbf{נשמר:} אינטליגנציה, עיבוד מידע, קשב, זיכרון לטווח קצר
    \item \textbf{נפגע:} יכולת ליצור זיכרונות חדשים (Anterograde Amnesia)
\end{itemize}
\end{exbox}

\begin{conceptbox}
\textbf{סוגי זיכרון ותלות בהיפוקמפוס:}

\begin{tabular}{|r|r|c|}
\hline
\rowcolor{tableheader} \textcolor{white}{\textbf{סוג זיכרון}} & \textcolor{white}{\textbf{תיאור}} & \textcolor{white}{\textbf{תלוי בהיפוקמפוס?}} \\
\hline
\rowcolor{tablerow1} זיכרון לטווח קצר & שניות-דקות, מוגבל & \xmark \\
\hline
\rowcolor{tablerow2} זיכרון דקלרטיבי & אירועים ועובדות & \cmark \\
\hline
\rowcolor{tablerow1} זיכרון פרוצדורלי & מוטורי (רכיבה על אופניים) & \xmark \\
\hline
\rowcolor{tablerow2} התניות & התניית פחד & \xmark{} (אמיגדלה) \\
\hline
\end{tabular}
\end{conceptbox}

\subsection{ניסויי Split-Brain (רוג'ר ספרי)}

\begin{exbox}
\textbf{דוגמה למבחן:}

טל עברה ניתוח Split-Brain. הוצגה לה "מזלג" בשדה ימין ו"כדור" בשדה שמאל.

\begin{itemize}
    \item אם יבקשו ממנה להרים ביד \textbf{ימין} $\rightarrow$ תרים \textbf{מזלג}
    \item אם יבקשו ממנה \textbf{לומר} מה ראתה $\rightarrow$ תגיד \textbf{"מזלג"}
    \item אם יבקשו ממנה להרים ביד \textbf{שמאל} $\rightarrow$ תרים \textbf{כדור}
\end{itemize}

\textbf{הסבר:} שדה ימין $\rightarrow$ המיספרה שמאל (שפה). שדה שמאל $\rightarrow$ המיספרה ימין (ללא שפה, שולטת ביד שמאל).
\end{exbox}

\subsection{שיטות לחקר המוח}

\begin{tabular}{|r|r|c|}
\hline
\rowcolor{tableheader} \textcolor{white}{\textbf{שיטה}} & \textcolor{white}{\textbf{מה עושה}} & \textcolor{white}{\textbf{מודדת או משפיעה?}} \\
\hline
\rowcolor{tablerow1} fMRI & מודדת שינויים בצריכת חמצן & מודדת \\
\hline
\rowcolor{tablerow2} EEG & מודדת פעילות חשמלית & מודדת \\
\hline
\rowcolor{tablerow1} TMS & משפיעה על פעילות באמצעות שדה מגנטי & \textbf{משפיעה!} \\
\hline
\end{tabular}

\begin{warnbox}
\textbf{TMS - גרייה מגנטית:} TMS (Transcranial Magnetic Stimulation) משתמשת בשדה מגנטי כדי \textbf{להשפיע} על פעילות נוירונים במוח - לא רק למדוד!
\end{warnbox}

\subsection{הייצוג הסומטוסנסורי}

\begin{notebox}
\textbf{חשוב למבחן:} הייצוג במוח נקבע לפי \textbf{רגישות}, לא לפי גודל פיזי!

לשפתיים יש שטח ייצוג \textbf{גדול יותר} מהגב - כי השפתיים \textbf{רגישות יותר} למגע.
\end{notebox}

\subsection{טבלת סיכום - חוקרי המוח}

\begin{tabular}{|r|r|r|}
\hline
\rowcolor{tableheader} \textcolor{white}{\textbf{חוקר}} & \textcolor{white}{\textbf{תרומה עיקרית}} & \textcolor{white}{\textbf{שנה}} \\
\hline
\rowcolor{tablerow1} רנה דקארט & דואליזם, בלוטת האצטרובל & המאה ה-17 \\
\hline
\rowcolor{tablerow2} פרנץ גאל & פרנולוגיה (נדחתה) & המאה ה-19 \\
\hline
\rowcolor{tablerow1} פול ברוקה & אזור ברוקה - הפקת שפה & 1861 \\
\hline
\rowcolor{tablerow2} קרל ורניקה & אזור ורניקה - הבנת שפה & 1874 \\
\hline
\rowcolor{tablerow1} רוג'ר ספרי & ניסויי Split-Brain & נובל 1981 \\
\hline
\rowcolor{tablerow2} ברנדה מילנר & מקרה H.M, זיכרון פרוצדורלי & שנות ה-50 \\
\hline
\end{tabular}

\newpage
% ============================================
\section{יחידה 5 - פסיכולוגיה התפתחותית}
% ============================================

\subsection{התקופה הקריטית}

\begin{defbox}
\textbf{הגדרה:} התקופה שבה החשיפה לסביבה אמורה להתרחש להתפתחות מלאה.
\end{defbox}

\begin{exbox}
\textbf{דוגמה - ג'יני:} ילדה שלא נחשפה לשפה עד גיל 13 - לא הצליחה ללמוד מעבר לרמה בסיסית.
\end{exbox}

\subsection{איך חוקרים מיינד של תינוקות?}

\begin{itemize}
    \item \textbf{קצב דופק} - שינוי בדופק לגירוי
    \item \textbf{תדירות מציצה} - שינוי בקצב מציצת מוצץ
    \item \textbf{זמן הסתכלות} - כמה זמן מסתכלים על גירוי (הנפוץ ביותר)
\end{itemize}

\subsection{Perceptual Narrowing}

\begin{conceptbox}
\begin{itemize}
    \item \textbf{עד 6 חודשים}: מבחינים בכל הפונמות בעולם ופנים מכל הגזעים
    \item \textbf{מגיל 9 חודשים}: מבחינים רק בפונמות של שפת האם ובפנים מוכרים
\end{itemize}
\end{conceptbox}

\subsubsection{מחקר פטרישיה קול (Patricia Kuhl)}

\begin{exbox}
\textbf{שאלה:} האם אפשר למנוע את ה-Perceptual Narrowing?

\textbf{ניסוי:} תינוקות אמריקאים (9 חודשים) נחשפו למנדרינית:

\begin{tabular}{|r|r|}
\hline
\rowcolor{tableheader} \textcolor{white}{\textbf{תנאי}} & \textcolor{white}{\textbf{תוצאה}} \\
\hline
\rowcolor{tablerow1} מפגש אישי עם דובר מנדרין & \textbf{שמירה} על יכולת ההבחנה \\
\hline
\rowcolor{tablerow2} צפייה בטלוויזיה בדובר מנדרין & \textbf{אין שמירה} - Perceptual Narrowing התרחש \\
\hline
\end{tabular}

\textbf{מסקנה:} \textbf{אינטראקציה חברתית} היא קריטית ללמידת שפה!
\end{exbox}

\subsection{צורות קניזסה (Kanizsa Figures)}

\begin{exbox}
\textbf{מה זה?} צורות שבהן אנחנו "רואים" קווים וצורות שלא קיימים פיזית - כמו משולש לבן שנראה "צף" מעל עיגולים שחורים.

\textbf{ניסוי עם תינוקות:}
\begin{itemize}
    \item הרגילו תינוקות לצורת קניזסה (ריבוע מדומה)
    \item הציגו צורה עם אותם חלקים אבל ללא אשליה
    \item מדדו זמן הסתכלות
\end{itemize}

\textbf{מסקנה:} כבר בגיל \textbf{חודשיים} תינוקות תופסים את המשולש המדומה!
\end{exbox}

\subsection{תיאוריית ההתפתחות של פיאז'ה}

\begin{tabular}{|r|c|r|}
\hline
\rowcolor{tableheader} \textcolor{white}{\textbf{שלב}} & \textcolor{white}{\textbf{גיל}} & \textcolor{white}{\textbf{מאפיינים}} \\
\hline
\rowcolor{tablerow1} סנסורי-מוטורי & 0-2 & קביעות אובייקט \\
\hline
\rowcolor{tablerow2} פרה-אופרציונלי & 2-7 & אגוצנטריות, שפה סימבולית \\
\hline
\rowcolor{tablerow1} אופרציונלי קונקרטי & 7-10 & הבנת שימור \\
\hline
\rowcolor{tablerow2} אופרציונלי פורמלי & 10+ & חשיבה אבסטרקטית \\
\hline
\end{tabular}

\subsection{Theory of Mind}

\begin{defbox}
\textbf{הגדרה:} היכולת להבין שלאנשים אחרים יש כוונות, אמונות ורצונות שונים משלי.

מתפתחת סביב גיל 4-5 (נבדקת במשימת False Belief).
\end{defbox}

\newpage
% ============================================
\section{יחידה 6 - פסיכולוגיה חברתית}
% ============================================

\subsection{השפעה על ביצועים}

\begin{tabular}{|r|r|}
\hline
\rowcolor{tableheader} \textcolor{white}{\textbf{תופעה}} & \textcolor{white}{\textbf{מתי}} \\
\hline
\rowcolor{tablerow1} הקלה חברתית & משימה קלה - ביצוע משתפר \\
\hline
\rowcolor{tablerow2} עכבה חברתית & משימה קשה - ביצוע נפגע \\
\hline
\end{tabular}

\subsubsection{הניסוי של רוברט זאיונק (על ג'וקים!)}

\begin{exbox}
\begin{itemize}
    \item \textbf{משימה קלה} (מסלול ישר): ביצוע טוב יותר בנוכחות אחרים
    \item \textbf{משימה קשה} (מסלול צלב): ביצוע טוב יותר לבד
\end{itemize}

\textbf{ההסבר:} עוררות ומוטיבציה עוזרות במשימות קלות אך מפריעות במשימות קשות.

\textbf{המשמעות:} התופעה קיימת גם בג'וקים ובחולדות - זו תופעה בסיסית ואוטומטית שלא דורשת קוגניציה גבוהה!
\end{exbox}

\subsubsection{האם הנוכחות או המבט?}

\begin{tabular}{|r|r|}
\hline
\rowcolor{tableheader} \textcolor{white}{\textbf{תנאי}} & \textcolor{white}{\textbf{ביצוע}} \\
\hline
\rowcolor{tablerow1} לבד & בסיסי \\
\hline
\rowcolor{tablerow2} נוכחות אחרים עם עיניים \textbf{מכוסות} & כמו לבד! \\
\hline
\rowcolor{tablerow1} נוכחות אחרים עם עיניים \textbf{פתוחות} & השפעה על הביצוע \\
\hline
\end{tabular}

\begin{notebox}
\textbf{מסקנה:} לא הנוכחות עצמה משפיעה, אלא \textbf{המבט} - העובדה שאנשים מתבוננים ויכולים להעריך!
\end{notebox}

\subsection{קונפורמיזם - הניסוי של סלומון אש}

\begin{exbox}
\textbf{הניסוי:} השוואת אורך קווים (תשובה אובייקטיבית!) כאשר משתפי פעולה עונים תשובות שגויות.

\textbf{תוצאות:}
\begin{itemize}
    \item 76\% ענו לפחות תשובה אחת שגויה
    \item 50\% ענו שגוי בלפחות מחצית מהסבבים
\end{itemize}
\end{exbox}

\begin{notebox}
\textbf{חשוב:} מספיק שאדם אחד בקבוצה יאמר תשובה שונה - והנבדק יאמר את התשובה הנכונה.
\end{notebox}

\subsection{אפקט הצופה מהצד (Bystander Effect)}

ככל שיש יותר אנשים - פחות סיכוי לעזור.

\subsubsection{המקרה של קיטי ג'נובס (1964)}

\begin{warnbox}
\textbf{הסיפור:}
\begin{itemize}
    \item ניו-יורק, 3 לפנות בוקר
    \item קיטי ג'נובס הותקפה בדרכה הביתה
    \item \textbf{ההתקפה נמשכה חצי שעה} - היא זעקה לעזרה
    \item לפי הדיווחים - 38 שכנים שמעו או ראו ואף אחד לא התערב
\end{itemize}

מקרה זה הוביל לחקר "אפקט הצופה מהצד".
\end{warnbox}

\subsubsection{ניסוי העשן (Latané \& Darley)}

\begin{exbox}
נבדקים ממלאים שאלון בחדר. פתאום עשן מתחיל להיכנס מתחת לדלת!

\begin{tabular}{|r|r|r|}
\hline
\rowcolor{tableheader} \textcolor{white}{\textbf{תנאי}} & \textcolor{white}{\textbf{דקה 2}} & \textcolor{white}{\textbf{דקה 6}} \\
\hline
\rowcolor{tablerow1} \textbf{לבד} & 50\% יצאו & 75\% יצאו \\
\hline
\rowcolor{tablerow2} \textbf{עם הרבה אנשים} & - & \textbf{0\% יצאו!} \\
\hline
\end{tabular}

\textbf{מסקנה:} נוכחות אחרים מונעת פעולה - גם כשיש סכנה אמיתית!
\end{exbox}

\textbf{הסברים:}
\begin{enumerate}
    \item \textbf{Pluralistic Ignorance} - "אם אחרים לא עוזרים, המצב לא מסוכן"
    \item \textbf{פיזור אחריות} - "מישהו אחר יטפל"
    \item \textbf{Audience Inhibition} - חשש מלהביך את עצמי
\end{enumerate}

\subsection{ניסוי מילגרם (1963)}

\begin{warnbox}
\textbf{תוצאות מזעזעות:} 60\%+ הסכימו לתת שוקים מסוכנים לאדם אחר כשנאמר להם לעשות זאת.
\end{warnbox}

\begin{defbox}
\textbf{טעות הייחוס בסיסית:} נטייה לייחס התנהגות של אחרים לאישיותם, ולא למצב שבו הם נמצאים.
\end{defbox}

\subsection{Stereotype Threat}

\begin{defbox}
\textbf{הגדרה:} חשש שביצועים יאוששו סטריאוטיפ שלילי.

\textbf{דוגמה:} נשים במתמטיקה - כשמזכירים הבדלים מגדריים, הביצוע יורד.
\end{defbox}

\subsection{טבלת סיכום - ניסויי פסיכולוגיה חברתית}

\begin{tabular}{|r|r|r|}
\hline
\rowcolor{tableheader} \textcolor{white}{\textbf{חוקר}} & \textcolor{white}{\textbf{ניסוי/תופעה}} & \textcolor{white}{\textbf{ממצא מרכזי}} \\
\hline
\rowcolor{tablerow1} נורמן טריפלט (1898) & גלגול חכה & ביצוע טוב יותר בנוכחות אחרים \\
\hline
\rowcolor{tablerow2} רוברט זאיונק & ג'וקים במסלול & קושי המשימה קובע אם נוכחות עוזרת \\
\hline
\rowcolor{tablerow1} מוזאפר שריף & אפקט אוטוקינטי & השפעה אינפורמטיבית \\
\hline
\rowcolor{tablerow2} סלומון אש & אורך קווים & קונפורמיזם גם בשיפוט אובייקטיבי \\
\hline
\rowcolor{tablerow1} סטנלי מילגרם & שוקים חשמליים & 60\%+ צייתו לסמכות \\
\hline
\rowcolor{tablerow2} לטנה ודארלי & אפקט הצופה & יותר אנשים = פחות עוזרים \\
\hline
\end{tabular}

\begin{warnbox}
\textbf{מה משותף למילגרם ואש?}

בשני הניסויים: \textbf{נוכחות של אדם שמורד/מציג דעה שונה - צמצמה את ההשפעה החברתית!}
\end{warnbox}

\newpage
% ============================================
\section{יחידה 7 - הבדלים בינאישיים}
% ============================================

\subsection{מבחני אינטליגנציה}

\begin{tabular}{|r|r|}
\hline
\rowcolor{tableheader} \textcolor{white}{\textbf{מאפיין}} & \textcolor{white}{\textbf{הסבר}} \\
\hline
\rowcolor{tablerow1} מהימנות & תוצאה דומה במדידות חוזרות \\
\hline
\rowcolor{tablerow2} תוקף & מודד את מה שאמור למדוד \\
\hline
\end{tabular}

\subsubsection{פרנסיס גלטון - ההתחלה}

\begin{exbox}
\textbf{גלטון (סוף המאה ה-19):}
\begin{itemize}
    \item הגדיר אינטליגנציה ככישורים \textbf{סנסוריים ומוטוריים}
    \item בנה מעבדה במוזיאון בלונדון
    \item בדק כ-10,000 אנשים!
\end{itemize}

\textbf{הבעיה:} המבחנים היו \textbf{מהימנים} אבל \textbf{לא תקפים} - אנשים "אינטליגנטים" לא קיבלו ציונים גבוהים יותר.

\textbf{תרומתו:} פיתח את מדד ה\textbf{מתאם} ומחקרי \textbf{תאומים}.
\end{exbox}

\subsection{מדידת מתאם - גרף פיזור (Scatter Plot)}

\begin{conceptbox}
\textbf{מתאם} נע בין 0 (אין קשר) ל-1 (קשר מלא):
\begin{itemize}
    \item \textbf{מתאם 0} = אם יודעים ציון של תאום אחד, לא יודעים את ציון השני
    \item \textbf{מתאם 1} = אם יודעים ציון של תאום אחד, יודעים בדיוק את ציון השני
\end{itemize}
\end{conceptbox}

\subsection{תיאוריית G (ספירמן)}

\begin{itemize}
    \item \textbf{G} = יכולת כללית משותפת לכל היכולות
    \item \textbf{S} = יכולות ספציפיות שאינן קשורות זו לזו
\end{itemize}

\subsection{חמשת הגדולים (Big 5)}

\begin{conceptbox}
\begin{tabular}{|r|r|}
\hline
\rowcolor{tableheader} \textcolor{white}{\textbf{תכונה}} & \textcolor{white}{\textbf{מאפיינים}} \\
\hline
\rowcolor{tablerow1} מוחצנות (Extroversion) & חברותיות, דברנות, פעלתנות \\
\hline
\rowcolor{tablerow2} יציבות רגשית (Neuroticism) & חרדה, דיכאון, עצבנות \\
\hline
\rowcolor{tablerow1} נועם הליכות (Agreeableness) & אדיבות, אמון, מזג נוח \\
\hline
\rowcolor{tablerow2} מצפוניות (Conscientiousness) & אחריות, יסודיות, משמעת עצמית \\
\hline
\rowcolor{tablerow1} פתיחות (Openness) & דמיון, סקרנות, אופקים רחבים \\
\hline
\end{tabular}
\end{conceptbox}

\subsection{מבחן המרשמלו (וולטר מישל)}

\begin{exbox}
\textbf{הניסוי:} ילדים יכולים לאכול מרשמלו עכשיו או לחכות ולקבל עוד אחד.

\textbf{מעקב לאורך שנים:} ילדים שהצליחו לחכות הפכו לבני נוער עם יכולת ריכוז גבוהה יותר, שליטה עצמית, וציונים גבוהים יותר.
\end{exbox}

\subsection{פרשנות מחקרי תאומים}

\begin{tabular}{|r|r|}
\hline
\rowcolor{tableheader} \textcolor{white}{\textbf{ממצא}} & \textcolor{white}{\textbf{פרשנות}} \\
\hline
\rowcolor{tablerow1} MZ $>$ DZ & תורשה משפיעה על התכונה \\
\hline
\rowcolor{tablerow2} MZ יחד $>$ MZ נפרד & סביבה משותפת משפיעה \\
\hline
\rowcolor{tablerow1} MZ $<$ 100\% & סביבה משפיעה (לא רק גנים) \\
\hline
\rowcolor{tablerow2} MZ נפרד דומים מאוד & תורשה חזקה מאוד \\
\hline
\end{tabular}

\begin{notebox}
\textbf{הבדל בין אינטליגנציה לאישיות:}

\textbf{אינטליגנציה:} השפעת \textbf{תורשה} חזקה יותר מסביבה.

\textbf{אישיות:} השפעת \textbf{סביבה לא-משותפת} (חברים, חוויות אישיות) חזקה יותר מסביבה משותפת.
\end{notebox}

\newpage
% ============================================
\section{יחידה 8 - פסיכופתולוגיה}
% ============================================

\subsection{ה-DSM}

\begin{defbox}
\textbf{DSM} = Diagnostic and Statistical Manual of Mental Disorders

ספר שמאגד את כל ההפרעות המנטליות והסימפטומים שמאפיינים אותן.
\end{defbox}

\textbf{קטגוריות הפרעות:}
\begin{itemize}
    \item הפרעות נוירו-התפתחותיות (אוטיזם, פיגור)
    \item סכיזופרניה
    \item הפרעות דיכאון
    \item הפרעות חרדה
    \item OCD
    \item הפרעות טראומה
    \item התמכרויות
    \item הפרעות אישיות
\end{itemize}

\subsection{סוגי טיפול}

\subsubsection{היסטוריה: לובוטומיה}

\begin{warnbox}
\textbf{לובוטומיה (Lobotomy)} - טיפול היסטורי שנזנח.

\textbf{מה זה?} ניתוק של הקורטקס הפרה-פרונטלי מחלקי מוח אחרים.

\textbf{שימוש:} בחצי הראשון של המאה ה-20 לטיפול בסכיזופרניה.

\textbf{תוצאות:}
\begin{itemize}
    \item גרמה לרגיעה
    \item \textbf{אבל} פגעה קשות ביכולות קוגניטיביות ורגשיות
\end{itemize}

\textbf{מה קרה?} נזנחה עם התגלית התרופות האנטי-פסיכוטיות בשנות ה-50. התרופה הראשונה אפילו נקראה "Pharmacological Lobotomy" כי יצרה אפקט דומה!
\end{warnbox}

\subsubsection{הטיפול הביולוגי}

\begin{tabular}{|r|r|}
\hline
\rowcolor{tableheader} \textcolor{white}{\textbf{סוג}} & \textcolor{white}{\textbf{שימוש}} \\
\hline
\rowcolor{tablerow1} ליתיום & הפרעה דו-קוטבית \\
\hline
\rowcolor{tablerow2} אנטי-פסיכוטיות & סכיזופרניה \\
\hline
\rowcolor{tablerow1} מרגיעות & חרדה \\
\hline
\rowcolor{tablerow2} נוגדות דיכאון & דיכאון (פרוזק) \\
\hline
\end{tabular}

\begin{notebox}
\textbf{על תרופות נוגדות דיכאון:}

מחקרים מראים שהן יעילות בעיקר למקרים \textbf{קשים} של דיכאון, אבל לא ברור עד כמה יעילות לדיכאון קל-בינוני לעומת פלסבו.
\end{notebox}

\textbf{טיפולים נוספים:}
\begin{itemize}
    \item \textbf{ECT} - שוקים חשמליים (לדיכאון קשה)
    \item \textbf{TMS} - גרייה מגנטית (פותחה גרסה בישראל)
\end{itemize}

\subsubsection{הטיפול הפסיכודינמי}

המקור לפתולוגיה: קונפליקט בין מודע ללא מודע.

\subsubsection{הטיפול הקוגניטיבי-התנהגותי (CBT)}

\begin{conceptbox}
\begin{itemize}
    \item \textbf{החלק ההתנהגותי (וולפה)}: הקהיה שיטתית - חשיפה הדרגתית לגירויים מפחידים
    \item \textbf{החלק הקוגניטיבי (אהרון בק)}: זיהוי הטיות קוגניטיביות ובחינת תוקפן
\end{itemize}
\end{conceptbox}

\begin{summarybox}
\textbf{מאפייני CBT:}
\begin{itemize}
    \item מטרות מוגדרות
    \item טיפול קצר וממוקד
    \item מטפל אקטיבי ומכוון
    \item שיעורי בית
    \item Psycho-education
\end{itemize}
\end{summarybox}

\subsubsection{השוואה: CBT לעומת ACT}

\begin{tabular}{|r|r|r|}
\hline
\rowcolor{tableheader} \textcolor{white}{\textbf{היבט}} & \textcolor{white}{\textbf{CBT}} & \textcolor{white}{\textbf{ACT}} \\
\hline
\rowcolor{tablerow1} מטרה & שינוי מחשבות שליליות & קבלת מחשבות ללא שיפוט \\
\hline
\rowcolor{tablerow2} יחס למחשבות & מחשבות יכולות להיות שגויות & מחשבות הן רק מחשבות \\
\hline
\rowcolor{tablerow1} טכניקה מרכזית & אתגור קוגניטיבי & Defusion (ניתוק) \\
\hline
\rowcolor{tablerow2} גישה לערכים & פחות מרכזי & מרכזי - פעולה לפי ערכים \\
\hline
\end{tabular}

\newpage
% ============================================
\section{הרצאות אורח}
% ============================================

\subsection{מדע התודעה (Consciousness Science)}

\begin{defbox}
\textbf{תודעה} היא החוויה הסובייקטיבית שלנו - "איך זה להיות" משהו.
\end{defbox}

\subsubsection{הבעיה הקשה (The Hard Problem)}

\begin{warnbox}
\textbf{דייוויד צ'אלמרס (1995):} מדוע עיבוד מידע במוח מלווה בחוויה סובייקטיבית?

לדוגמה: מדוע כשאנחנו רואים אדום, יש לנו \textbf{חוויה} של אדום, ולא רק עיבוד אינפורמציה?
\end{warnbox}

\subsubsection{הבחנות חשובות}

\begin{tabular}{|r|r|}
\hline
\rowcolor{tableheader} \textcolor{white}{\textbf{הבחנה}} & \textcolor{white}{\textbf{הסבר}} \\
\hline
\rowcolor{tablerow1} רמת תודעה (Arousal) & האם ער או ישן? כמה ערני? \\
\hline
\rowcolor{tablerow2} תוכן תודעה (Awareness) & מה התוכן של החוויה? \\
\hline
\rowcolor{tablerow1} תודעה פנומנלית & החוויה הסובייקטיבית עצמה \\
\hline
\rowcolor{tablerow2} תודעה אקסס (Access) & מידע זמין לדיווח ופעולה \\
\hline
\end{tabular}

\subsubsection{NCC - Neural Correlates of Consciousness}

\begin{defbox}
\textbf{NCC} = המתאמים העצביים של התודעה - מה קורה במוח כשיש חוויה מודעת?
\end{defbox}

\subsubsection{תיאוריות עיקריות של תודעה}

\begin{tabular}{|r|r|r|}
\hline
\rowcolor{tableheader} \textcolor{white}{\textbf{תיאוריה}} & \textcolor{white}{\textbf{קיצור}} & \textcolor{white}{\textbf{רעיון מרכזי}} \\
\hline
\rowcolor{tablerow1} Global Workspace Theory & GWT & מידע מודע "משודר" לכל המוח \\
\hline
\rowcolor{tablerow2} Integrated Information Theory & IIT & תודעה = אינטגרציה של מידע \\
\hline
\rowcolor{tablerow1} Higher-Order Thought & HOT & תודעה = מחשבות על מחשבות \\
\hline
\rowcolor{tablerow2} Predictive Processing & PP/RPT & המוח חוזה ומשווה לקלט \\
\hline
\end{tabular}

\subsubsection{חוקרי תודעה}

\begin{tabular}{|r|r|}
\hline
\rowcolor{tableheader} \textcolor{white}{\textbf{חוקר}} & \textcolor{white}{\textbf{תרומה עיקרית}} \\
\hline
\rowcolor{tablerow1} דייוויד צ'אלמרס & הבעיה הקשה של התודעה \\
\hline
\rowcolor{tablerow2} תומס נייגל & "מה זה להיות עטלף?" \\
\hline
\rowcolor{tablerow1} ג'ון סרל & Chinese Room, תודעה ביולוגית \\
\hline
\rowcolor{tablerow2} אניל סת' & תודעה כהזיה מבוקרת \\
\hline
\end{tabular}

\newpage
\subsection{פסיכולוגיה חברתית יישומית}

\textbf{מרצה:} פרופ' נורית שנבל

\subsubsection{Nudge - דחיפה עדינה}

\begin{defbox}
\textbf{Nudge} (Thaler \& Sunstein) = שינוי קטן בסביבת הבחירה שמשפיע על החלטות, בלי לאסור אפשרויות.
\end{defbox}

\begin{exbox}
\textbf{ניסוי חיסון טטנוס (Leventhal):}

\textbf{קבוצה א':} קיבלו מידע מפחיד על טטנוס $\rightarrow$ מעט התחסנו

\textbf{קבוצה ב':} קיבלו מידע + מפה למרפאה + שעות פעילות $\rightarrow$ \textbf{הרבה יותר} התחסנו

\textbf{מסקנה:} מידע לבד לא מספיק - צריך להקל על הפעולה!
\end{exbox}

\subsubsection{השערת המגע (Contact Hypothesis)}

\begin{conceptbox}
\textbf{אולפורט (1954):} מגע בין קבוצות יכול להפחית דעות קדומות - אבל רק בתנאים מסוימים!

\textbf{ארבעת התנאים:}
\begin{enumerate}
    \item \textbf{מעמד שווה} בין הקבוצות
    \item \textbf{מטרות משותפות}
    \item \textbf{תמיכה מוסדית}
    \item \textbf{קשרים אישיים} (לא שטחיים)
\end{enumerate}
\end{conceptbox}

\subsubsection{ניסוי השומר הטוב}

\begin{exbox}
\textbf{Darley \& Batson:}

סטודנטים לתיאולוגיה הלכו לבניין אחר לתת הרצאה. בדרך: שתול שנראה זקוק לעזרה.

\begin{itemize}
    \item \textbf{ממהרים}: רק 10\% עזרו
    \item \textbf{לא ממהרים}: 63\% עזרו
    \item \textbf{נושא ההרצאה לא השפיע!}
\end{itemize}

\textbf{מסקנה:} הסיטואציה משפיעה יותר מהאישיות!
\end{exbox}

\newpage
\subsection{פסיכותרפיה מבוססת ראיות}

\textbf{מרצה:} ד"ר מיכל כהן

\begin{defbox}
\textbf{Evidence-Based Psychotherapy} = טיפול שהוכח במחקר מדעי כיעיל ובטוח.
\end{defbox}

\subsubsection{חמש השאלות של המחקר הקליני}

\begin{tabular}{|r|r|r|}
\hline
\rowcolor{tableheader} \textcolor{white}{\textbf{\#}} & \textcolor{white}{\textbf{שאלה}} & \textcolor{white}{\textbf{מה בודקת}} \\
\hline
\rowcolor{tablerow1} 1 & Horse-race & מה הטיפול הטוב ביותר? \\
\hline
\rowcolor{tablerow2} 2 & Moderation & מה עובד למי? (באילו תנאים) \\
\hline
\rowcolor{tablerow1} 3 & Mediation & מה המנגנון הפעיל? \\
\hline
\rowcolor{tablerow2} 4 & Safety & האם הטיפול בטוח? \\
\hline
\rowcolor{tablerow1} 5 & Translation & האם ניתן ליישום בשטח? \\
\hline
\end{tabular}

\subsubsection{טיפולים בנדודי שינה אצל תינוקות}

\begin{tabular}{|r|r|}
\hline
\rowcolor{tableheader} \textcolor{white}{\textbf{שיטה}} & \textcolor{white}{\textbf{תיאור}} \\
\hline
\rowcolor{tablerow1} Extinction & לא להגיב לבכי כלל \\
\hline
\rowcolor{tablerow2} Modified Extinction & לבדוק במרווחים קבועים \\
\hline
\rowcolor{tablerow1} Camping-out & להיות בחדר ולהתרחק בהדרגה \\
\hline
\rowcolor{tablerow2} Parental Presence & להישאר עם הילד \\
\hline
\end{tabular}

\subsubsection{Huggy-Puppy Intervention}

\begin{exbox}
\textbf{פרופ' אבי שדה:}

\textbf{הרעיון:} ילד שעבר טראומה מקבל בובה ו"מטפל" בה.

\textbf{למה זה עובד:}
\begin{itemize}
    \item הילד עובר מ"קורבן" ל"מטפל"
    \item תחושת שליטה ואחריות
    \item הקרנה של רגשות על הבובה
\end{itemize}

\textbf{יעילות:} Cohen's d = 0.94-0.99 (גבוהה מאוד!)
\end{exbox}

% ============================================
\vfill
\begin{center}
\rule{0.5\textwidth}{0.5pt}\\[0.3cm]
\textit{סיכום זה נוצר באמצעות Claude AI}\\
\url{https://github.com/orinlevi/Intro_to_psychlogy}
\end{center}

\end{document}
