% מבוא לפסיכולוגיה - סיכום מקיף
% XeLaTeX with Hebrew support
\documentclass[11pt,a4paper]{article}

% Hebrew and RTL support
\usepackage{fontspec}
\usepackage{polyglossia}
\setmainlanguage{hebrew}
\setotherlanguage{english}
\setmainfont{David CLM}[Script=Hebrew]
\newfontfamily\hebrewfont{David CLM}[Script=Hebrew]
\newfontfamily\englishfont{Times New Roman}

% Page layout
\usepackage[margin=2cm]{geometry}
\usepackage{fancyhdr}
\pagestyle{fancy}
\fancyhf{}
\fancyhead[R]{\leftmark}
\fancyhead[L]{מבוא לפסיכולוגיה}
\fancyfoot[C]{\thepage}

% Colors and boxes
\usepackage{xcolor}
\usepackage{tcolorbox}
\tcbuselibrary{skins,breakable}

\definecolor{tealcolor}{RGB}{0,128,128}
\definecolor{cyancolor}{RGB}{0,188,212}
\definecolor{lightgray}{RGB}{245,245,245}

% Tables
\usepackage{booktabs}
\usepackage{array}
\usepackage{tabularx}

% Lists and formatting
\usepackage{enumitem}
\setlist[itemize]{nosep, leftmargin=*}
\setlist[enumerate]{nosep, leftmargin=*}

% Hyperlinks
\usepackage{hyperref}
\hypersetup{
    colorlinks=true,
    linkcolor=tealcolor,
    urlcolor=cyancolor
}

% Section formatting
\usepackage{titlesec}
\titleformat{\section}{\Large\bfseries\color{tealcolor}}{\thesection}{1em}{}
\titleformat{\subsection}{\large\bfseries\color{cyancolor}}{\thesubsection}{1em}{}

% Custom boxes
\newtcolorbox{infobox}[1][]{
    colback=lightgray,
    colframe=tealcolor,
    fonttitle=\bfseries,
    title=#1,
    breakable
}

\newtcolorbox{warnbox}[1][]{
    colback=yellow!10,
    colframe=orange!80!black,
    fonttitle=\bfseries,
    title=#1,
    breakable
}

\newtcolorbox{tipbox}[1][]{
    colback=green!10,
    colframe=green!60!black,
    fonttitle=\bfseries,
    title=#1,
    breakable
}

% Title
\title{
    \vspace{-1cm}
    {\Huge\bfseries מבוא לפסיכולוגיה}\\[0.5cm]
    {\Large סיכום מקיף לקורס}\\[0.3cm]
    {\large אוניברסיטת תל אביב}
}
\author{נוצר על ידי Orin Levi}
\date{\today}

\begin{document}

\maketitle
\tableofcontents
\newpage

%====================================
\section{יחידה 1 - מהי פסיכולוגיה?}
%====================================

\subsection{הגדרות פסיכולוגיה}
\begin{itemize}
    \item \textbf{לפי ויקיפדיה}: פסיכולוגיה היא מדע ההתנהגות והתהליכים המנטליים
    \item \textbf{לפי מילון מרים וובסטר}: פסיכולוגיה היא מדע המיינד וההתנהגות
    \item \textbf{לפי ספר הלימוד}: פסיכולוגיה היא המחקר המדעי של התנהגות ותהליכים מנטליים
\end{itemize}

\subsection{המדע האמפירי}
\begin{infobox}[מאפייני השיטה המדעית]
\begin{itemize}
    \item \textbf{תצפיות ומדידות אובייקטיביות} - הבסיס של המדע האמפירי
    \item \textbf{יכולת ניבוי} - היתרון המשמעותי של המדע האמפירי לעומת מקורות ידע לא מדעיים
    \item \textbf{תיאוריות זמניות} - המדע לא מקבל שום דבר כאמת נצחית
    \item \textbf{שיחזור (רפליקציה)} - חזרה על תוצאות במחקרים שונים
\end{itemize}
\end{infobox}

\subsection{דוגמה: טיפול ב-PTSD}
הפרעת PTSD היא הפרעה נפשית שמתפתחת כתוצאה מאירוע טראומטי שמאיים על הביטחון וגורם להרגשה של חוסר אונים.

\textbf{סימפטומים עיקריים:}
\begin{itemize}
    \item חוויה מחדש של הטראומה (סיוטים או מחשבות חודרניות)
    \item עלייה כללית בעוררות ובתגובות חרדתיות
    \item ניסיון להימנע מדברים שמזכירים את הטראומה
\end{itemize}

\begin{tabularx}{\textwidth}{|X|X|X|}
\hline
\textbf{טיפול} & \textbf{תיאור} & \textbf{יעילות} \\
\hline
רלקסציה & לימוד תרגילי כיווץ והרפיית שרירים & 40\% החלמה \\
\hline
PE (חשיפה מתמשכת) & חשיפה אינטנסיבית לאירוע הטראומטי & 90\% החלמה \\
\hline
\end{tabularx}

\subsection{מדע בסיסי לעומת מדע יישומי}
\begin{itemize}
    \item \textbf{מדע בסיסי}: מנסה להבין את המנגנון שעשוי להסביר תופעה בעולם
    \item \textbf{מדע יישומי}: מאפשר לפתח שיטות שישפרו את איכות החיים שלנו
\end{itemize}

\newpage
%====================================
\section{יחידה 2 - גישות בפסיכולוגיה: הביהביוריזם}
%====================================

\subsection{מבוא לביהביוריזם}
הזרם המרכזי ששלט בפסיכולוגיה בחלק הראשון של המאה ה-20.

\begin{warnbox}[הקופסה השחורה]
תהליכים מנטליים כמו זיכרון, חשיבה, שפה וקבלת החלטות - אי אפשר לראות ולכן (לטענת הביהביוריסטים) אי אפשר לחקור באופן מדעי.
\end{warnbox}

\subsection{איוון פבלוב - התניה קלאסית}
פבלוב היה פיזיולוג רוסי שחקר את מערכת העיכול וזכה בפרס נובל.

\begin{tabularx}{\textwidth}{|l|X|X|}
\hline
\textbf{מונח} & \textbf{הסבר} & \textbf{דוגמה} \\
\hline
גירוי בלתי מותנה & גירוי שיוצר תגובה אוטומטית & אוכל \\
\hline
תגובה בלתי מותנית & תגובה אוטומטית לגירוי & ריור \\
\hline
גירוי מותנה & גירוי ניטרלי שנקשר לגירוי בלתי מותנה & צליל \\
\hline
תגובה מותנית & תגובה לגירוי המותנה & ריור לצליל \\
\hline
\end{tabularx}

\subsection{תהליכי למידה}
\begin{enumerate}
    \item \textbf{רכישה (Acquisition)} - למידת הקשר בין גירוי מותנה לתגובה מותנית
    \item \textbf{הכחדה (Extinction)} - הפסקת הצימוד וירידה בתגובה
    \item \textbf{החלמה ספונטנית (Spontaneous Recovery)} - חזרת התגובה למחרת
\end{enumerate}

\subsection{ג'ון ווטסון - התניית פחד}
הניסוי של אלברט הקטן בדק האם ניתן לגרום לתינוק לפחד מחיה שאינו מפחד ממנה.

\subsection{ב.פ. סקינר - התניה אופרנטית}

\begin{tabularx}{\textwidth}{|l|l|X|X|}
\hline
\textbf{סוג} & \textbf{מטרה} & \textbf{מנגנון} & \textbf{דוגמה} \\
\hline
חיזוק חיובי & הגברת התנהגות & מתן דבר נעים & "אם תכין שיעורים תוכל לשחק" \\
\hline
חיזוק שלילי & הגברת התנהגות & הסרת דבר לא נעים & צפצוף שנפסק בחגירת חגורה \\
\hline
עונש חיובי & הפחתת התנהגות & מתן דבר לא נעים & קנס על חנייה \\
\hline
עונש שלילי & הפחתת התנהגות & מניעת דבר נעים & "אם תצעוק לא תקבל סוכריה" \\
\hline
\end{tabularx}

\newpage
%====================================
\section{יחידה 3 - קוגניציה}
%====================================

\subsection{המהפכה הקוגניטיבית}
\begin{itemize}
    \item \textbf{נועם חומסקי} - בלשן מ-MIT, טען על קיום מנגנון מולד לרכישת שפה (LAD)
    \item \textbf{ג'ורג' מילר} - חקר את גודל זיכרון העבודה ($7\pm2$ פריטים)
\end{itemize}

\begin{infobox}[שתי הטענות המרכזיות]
\begin{enumerate}
    \item מדע הפסיכולוגיה חייב ויכול לחקור באופן מדעי גם תהליכים מנטליים
    \item אי אפשר להסביר התנהגות רק על סמך למידה מהסביבה - יש מרכיב מולד
\end{enumerate}
\end{infobox}

\subsection{מנגנון הקשב}
היכולת לעבד מידע מוגבלת - צריך לבחור לאיזה מידע לתת עדיפות.

\textbf{שתי דרכים להפניית קשב:}
\begin{enumerate}
    \item \textbf{הגירוי עצמו} - גירויים בולטים ועוצמתיים ימשכו קשב
    \item \textbf{רצון/ציפיות} - קשב רצוני (תופעת מסיבת הקוקטייל)
\end{enumerate}

\subsection{ייצוג (Representation)}
המיינד מייצר ייצוג של העולם החיצוני שאינו זהה לגירוי עצמו.

\begin{tabularx}{\textwidth}{|l|X|X|}
\hline
\textbf{תהליך} & \textbf{מקור} & \textbf{דוגמה} \\
\hline
Bottom-up & הגירוי עצמו & מה שנכנס דרך החושים \\
\hline
Top-down & ידע, זיכרון, ציפיות & פרשנות שמשפיעה על הייצוג \\
\hline
\end{tabularx}

\subsection{זיכרון מובנה (Constructive Memory)}
\begin{tipbox}[הניסוי של אליזבת לופטוס (1974)]
נבדקים צפו בסרט תאונה ונשאלו על המהירות.

שאלה עם "נמחצה" הובילה להערכת מהירות גבוהה יותר + "זיכרון" של זכוכית שבורה שלא הייתה.

\textbf{השלכות}: לאופן ניסוח השאלה יש השפעה עצומה על דיווח העדים.
\end{tipbox}

\newpage
%====================================
\section{יחידה 4 - גוף ומוח}
%====================================

\subsection{הקשר בין גוף ונפש}
\begin{itemize}
    \item \textbf{רדוקציוניזם}: התיאור הביולוגי של תהליכים פסיכולוגיים
    \item \textbf{קשר דו-כיווני}: המוח יוצר מחשבות, והמחשבות משנות את המוח
\end{itemize}

\subsection{הנוירון}
\begin{tabularx}{\textwidth}{|l|X|}
\hline
\textbf{חלק} & \textbf{תפקיד} \\
\hline
דנדריטים & קבלת מידע מנוירונים אחרים \\
\hline
גוף התא & עיבוד המידע \\
\hline
אקסון & העברת הסיגנל החשמלי \\
\hline
כפתורים טרמינליים & שחרור נוירוטרנסמיטרים \\
\hline
\end{tabularx}

\subsection{אזורי מוח חשובים}
\begin{itemize}
    \item \textbf{אזור ברוקה} (פול ברוקה, 1861) - פגיעה גורמת לבעיות בהפקת שפה
    \item \textbf{אזור ורניקה} (קרל ורניקה, 1874) - פגיעה גורמת לדיבור שוטף אך חסר משמעות
\end{itemize}

\subsection{המקרה של H.M}
H.M עבר ניתוח להסרת ההיפוקמפוס משני הצדדים בגלל אפילפסיה.

\begin{tabularx}{\textwidth}{|l|X|c|}
\hline
\textbf{סוג זיכרון} & \textbf{תיאור} & \textbf{תלוי בהיפוקמפוס?} \\
\hline
זיכרון לטווח קצר & שניות-דקות, מוגבל & לא \\
\hline
זיכרון דקלרטיבי & אירועים ועובדות & כן \\
\hline
זיכרון פרוצדורלי & מוטורי (רכיבה על אופניים) & לא \\
\hline
התניות & התניית פחד & לא (אמיגדלה) \\
\hline
\end{tabularx}

\newpage
%====================================
\section{יחידה 5 - פסיכולוגיה התפתחותית}
%====================================

\subsection{התקופה הקריטית}
התקופה שבה החשיפה לסביבה אמורה להתרחש להתפתחות מלאה.

\begin{warnbox}[דוגמה: ג'יני]
ילדה שלא נחשפה לשפה עד גיל 13 - לא הצליחה ללמוד מעבר לרמה בסיסית.
\end{warnbox}

\subsection{איך חוקרים מיינד של תינוקות?}
\begin{itemize}
    \item \textbf{קצב דופק} - שינוי בדופק לגירוי
    \item \textbf{תדירות מציצה} - שינוי בקצב מציצת מוצץ
    \item \textbf{זמן הסתכלות} - כמה זמן מסתכלים על גירוי (הנפוץ ביותר)
\end{itemize}

\subsection{Perceptual Narrowing}
\begin{itemize}
    \item \textbf{עד 6 חודשים}: מבחינים בין כל הפונמות בעולם ופנים מכל הגזעים
    \item \textbf{מגיל 9 חודשים}: מבחינים רק בפונמות של שפת האם ובפנים מוכרים
\end{itemize}

\subsection{תיאוריית ההתפתחות של פיאז'ה}
\begin{tabularx}{\textwidth}{|l|c|X|}
\hline
\textbf{שלב} & \textbf{גיל} & \textbf{מאפיינים} \\
\hline
סנסורי-מוטורי & 0-2 & למידה דרך חושים ותנועה, קביעות אובייקט \\
\hline
פרה-אופרציונלי & 2-7 & שפה וחשיבה סימבולית, אגוצנטריות \\
\hline
אופרציונלי קונקרטי & 7-10 & הבנת שימור, נקודת מבט של אחר \\
\hline
אופרציונלי פורמלי & 10+ & חשיבה היפותטית ואבסטרקטית \\
\hline
\end{tabularx}

\subsection{Theory of Mind}
היכולת להבין שלאנשים אחרים יש כוונות, אמונות ורצונות שונים משלי.

מתפתחת סביב גיל 4-5 (נבדקת במשימת False Belief).

\newpage
%====================================
\section{יחידה 6 - פסיכולוגיה חברתית}
%====================================

\subsection{השפעה על ביצועים}
\begin{tabularx}{\textwidth}{|l|X|}
\hline
\textbf{תופעה} & \textbf{מתי} \\
\hline
הקלה חברתית & משימה קלה - ביצוע משתפר \\
\hline
עכבה חברתית & משימה קשה - ביצוע נפגע \\
\hline
\end{tabularx}

\subsection{קונפורמיזם - הניסוי של סלומון אש}
\begin{infobox}[תוצאות]
\begin{itemize}
    \item 76\% ענו לפחות תשובה אחת שגויה
    \item 50\% ענו שגוי בלפחות מחצית מהסבבים
    \item מספיק שאדם אחד בקבוצה יאמר תשובה שונה - והנבדק יאמר את התשובה הנכונה
\end{itemize}
\end{infobox}

\subsection{אפקט הצופה מהצד}
ככל שיש יותר אנשים - פחות סיכוי לעזור.

\textbf{הסברים:}
\begin{enumerate}
    \item \textbf{Pluralistic Ignorance} - "אם אחרים לא עוזרים, כנראה המצב לא מסוכן"
    \item \textbf{פיזור אחריות} - "מישהו אחר יטפל"
    \item \textbf{Audience Inhibition} - חשש מלהביך את עצמי
\end{enumerate}

\subsection{ניסוי מילגרם (1963)}
\begin{warnbox}[תוצאות מזעזעות]
60\%+ הסכימו לתת שוקים מסוכנים לאדם אחר כשנאמר להם לעשות זאת.

\textbf{טעות הייחוס בסיסית}: נטייה לייחס התנהגות של אחרים לאישיותם, ולא למצב.
\end{warnbox}

\subsection{Stereotype Threat}
חשש שביצועים יאוששו סטריאוטיפ שלילי.

\textbf{דוגמה}: נשים במתמטיקה - כשמזכירים הבדלים מגדריים, הביצוע יורד.

\newpage
%====================================
\section{יחידה 7 - הבדלים בינאישיים}
%====================================

\subsection{מבחני אינטליגנציה}
\begin{tabularx}{\textwidth}{|l|X|}
\hline
\textbf{מאפיין} & \textbf{הסבר} \\
\hline
מהימנות & דיוק - תוצאה דומה במדידות חוזרות \\
\hline
תוקף & מודד את מה שאמור למדוד \\
\hline
\end{tabularx}

\subsection{תיאוריית G (ספירמן)}
\begin{itemize}
    \item \textbf{G} = יכולת כללית משותפת לכל היכולות
    \item \textbf{S} = יכולות ספציפיות שאינן קשורות זו לזו
\end{itemize}

\subsection{השפעת תורשה וסביבה}
\begin{tipbox}[ממצאים]
לתורשה תרומה משמעותית יותר מסביבה להבדלים באינטליגנציה.

תאומים זהים שנפרדו מראים מתאם גבוה ב-IQ.
\end{tipbox}

\subsection{חמשת הגדולים (Big 5)}
\begin{tabularx}{\textwidth}{|l|X|}
\hline
\textbf{תכונה} & \textbf{מאפיינים} \\
\hline
מוחצנות (Extroversion) & חברותיות, דברנות, פעלתנות \\
\hline
יציבות רגשית (Neuroticism) & חרדה, דיכאון, עצבנות \\
\hline
נועם הליכות (Agreeableness) & אדיבות, אמון, מזג נוח \\
\hline
מצפוניות (Conscientiousness) & אחריות, יסודיות, משמעת עצמית \\
\hline
פתיחות (Openness) & דמיון, סקרנות, אופקים רחבים \\
\hline
\end{tabularx}

\subsection{מבחן המרשמלו (וולטר מישל)}
ילדים שהצליחו לדחות סיפוקים הפכו לבני נוער עם יכולת ריכוז, שליטה עצמית, וציונים גבוהים יותר.

\newpage
%====================================
\section{יחידה 8 - פסיכופתולוגיה}
%====================================

\subsection{מהי פסיכופתולוגיה?}
הפרעת נפש - מחלה שמקורה בהפרעה מוחית.

\textbf{שני אתגרים מרכזיים:}
\begin{enumerate}
    \item \textbf{אבחון} - האם אדם סובל מהפרעה ומאיזו
    \item \textbf{טיפול} - הצעת טיפול מתאים ואפקטיבי
\end{enumerate}

\subsection{ה-DSM}
\textbf{DSM} = Diagnostic and Statistical Manual of Mental Disorders

ספר שמאגד את כל ההפרעות המנטליות והסימפטומים שמאפיינים אותן.

\begin{infobox}[קטגוריות הפרעות]
\begin{itemize}
    \item הפרעות נוירו-התפתחותיות (אוטיזם, פיגור)
    \item סכיזופרניה
    \item הפרעות דיכאון
    \item הפרעות חרדה
    \item הפרעה אובססיבית קומפולסיבית
    \item הפרעות טראומה
    \item התמכרויות
    \item הפרעות אישיות
\end{itemize}
\end{infobox}

\subsection{סוגי טיפול}

\subsubsection{הטיפול הביולוגי}
\begin{tabularx}{\textwidth}{|l|X|}
\hline
\textbf{סוג} & \textbf{שימוש} \\
\hline
ליתיום & הפרעה דו-קוטבית \\
\hline
אנטי-פסיכוטיות & סכיזופרניה (הזיות, מחשבות שווא) \\
\hline
מרגיעות & חרדה \\
\hline
נוגדות דיכאון & דיכאון (פרוזק) \\
\hline
\end{tabularx}

\subsubsection{הטיפול הפסיכודינמי (מבוסס על פרויד)}
\begin{itemize}
    \item המקור לפתולוגיה: קונפליקט בין מודע ללא מודע
    \item טכניקות: אסוציאציות חופשיות, ניתוח חלומות, Transference
\end{itemize}

\subsubsection{הטיפול הקוגניטיבי-התנהגותי (CBT)}
\begin{itemize}
    \item \textbf{החלק ההתנהגותי (וולפה)}: הקהיה שיטתית - חשיפה הדרגתית לגירויים מפחידים
    \item \textbf{החלק הקוגניטיבי (אהרון בק)}: זיהוי הטיות קוגניטיביות ובחינת תוקפן
\end{itemize}

\begin{tipbox}[מאפייני CBT]
\begin{itemize}
    \item מטרות מוגדרות
    \item טיפול קצר וממוקד
    \item מטפל אקטיבי ומכוון
    \item שיעורי בית
    \item Psycho-education
\end{itemize}
\end{tipbox}

\subsection{הגל השלישי}
גישות חדשות כמו מיינדפולנס (קשיבות) ו-ACT - התבוננות על מחשבות מהצד וקבלתן.

%====================================
\vfill
\begin{center}
\rule{0.5\textwidth}{0.5pt}\\[0.3cm]
\textit{סיכום זה נוצר באמצעות Claude AI}\\
\url{https://github.com/orinlevi/intro_to_psychology_notes}
\end{center}

\end{document}
